\chapter{Elaine Specification}\label{chap:spec}

\TODO{Custom type declarations are in the language but not explained in this chapter yet.}

This chapter contains the detailed specification for Elaine: the syntax, semantics, type inference rules, and the functions provided by the standard library.

\section{Syntax Definition}\label{sec:syntax}

The Elaine syntax was designed to be relatively easy to parse. The grammar is white-space insensitive and most constructs are unambiguously identified with keywords at the start.

Based on the previous chapters, the \tok{elab} without an elaboration might be surprising. The use of that syntax is explained in \cref{chap:elabres}.

The full syntax definition is given in \cref{fig:syntax}. For convenience, we use several extensions to BNF:
\begin{itemize}
    \item tokens are written in \tok{monospace font}, this includes the tokens \tok{[]}, \tok{<>}, \tok{|}, and \tok{!}, which might be confused with the syntax of BNF,
    \item $\opt{p}$ indicates that the sort $p$ is optional,
    \item $\rep{p}$ indicates that the sort $p$ can be repeated zero or more times, and
    \item $\csep{p}$ indicates that the sort $p$ can be repeated zero or more times, separated by commas.
\end{itemize}

\begin{figure}[p]
\begin{align*}
    \text{program}\ p
        \IS & d \dots d\\
    \\
    \text{declaration}\ d
        \IS & \opt{\tok{pub}}\ \tok{mod}\ x\ \tok{\{} \rep{d} \tok{\}}\\
        \OR & \opt{\tok{pub}}\ \tok{use}\ x\tok{;}\\
        \OR & \opt{\tok{pub}}\ \tok{let}\ \opt{\tok{rec}}\ p\ \tok{=}\ e\tok{;}\\
        \OR & \opt{\tok{pub}}\ \tok{effect}\ \phi\ \tok{\{} \csep{s} \tok{\}}\\
        \OR & \opt{\tok{pub}}\ \tok{type}\ x\ \tok{\{} \csep{s} \tok{\}}\\
    \\
    \text{block}\ b
        \IS & \tok{\{}\ es\ \tok{\}}\\
    \text{expression list}\ es
        \IS & e\tok{;}\ es\\
        \OR & \tok{let}\ \opt{\tok{rec}}\ p\ \tok{=}\ e\tok{;}\ es\\
        \OR & e\\
    \text{expression}\ e
        \IS & x\\
        \OR & \tok{()} \OR \tok{true} \OR \tok{false} \OR number \OR string \\
        \OR & \tok{(} \csep{e} \tok{)} \\
        \OR & \tok{fn} \tok{(} \csep{p} \kw{)}\ [T]\ b \\
        \OR & \tok{if}\ e\ b\ \tok{else}\ b\\
        \OR & e\tok{(} \csep{e} \tok{)} \OR \phi\tok{(} \csep{e} \tok{)} \\
        \OR & \tok{handler}\ \tok{\{} \tok{return} \tok{(}x\tok{)}\ b \tok{,}\ \csep{o} \tok{\}}\\
        \OR & \tok{handle}\tok{[}e\tok{]}\ e \\
        \OR & \tok{elaboration}\ x\tok{!}\ \tok{->}\ \Delta\ \tok{\{} \csep{o} \tok{\}}\\
        \OR & \tok{elab}\tok{[} e \tok{]}\ e \OR \tok{elab}\ e \\
        \OR & es\\
    \\
    \text{annotatable variable}\ p
        \IS & x\ \tok{:}\ T \OR x\\
    \text{signature}\ s
        \IS & x \tok{(} \csep{T} \tok{)}\ T\\
    \text{effect clause}\ o
        \IS & x \tok{(} \csep{x} \tok{)}\ b\\
    \\
    \text{type}\ T
        \IS & \Delta\ \tau \OR \tau\\
    \text{value type}\ \tau
        \IS & x \\ 
        \OR & \tok{()} \OR \tok{Bool} \OR \tok{Int} \OR \tok{String}\\
        \OR & \tok{fn} \tok{(} \csep{T} \tok{)}\ T \\
        \OR & \tok{handler}\ x\ \tau\ \tau \\
        \OR & \tok{elaboration}\ x!\ \Delta \\
    \text{effect row}\ \Delta
        \IS & \tok{<} \csep{\phi} \opt{\tok{|} x} \tok{>} \\
    \text{effect}\ \phi \IS & x \OR x\tok{!}
\end{align*}
\caption{Syntax definition of Elaine}
\label{fig:syntax}
\end{figure}

\section{Effect row semantics}\label{sec:effectrows}

Elaine's type checker uses multisets to model effect rows, meaning that the row $\row{A, B, B, C}$ is represented by the multiset $\S{A, B, B, C}$. This yields a semantics where the multiplicity of effects is significant, but the order is not.


Since the effect row of a computation must match the effect row of the context in which it is used, the effect row of the computation is an overapproximation of the effects that are necessary. Therefore, we should allow effect row polymorphism, so that the same expression can be used within multiple contexts.

Effect row polymorphism is enabled via the \emph{row tail}, which is denoted with the \el{|} symbol followed by an identifier.

The $|$ symbol signifies extension of the effect row with another (possibly arbitrary) effect row. We determine compatibility between effect rows by unifying them. That is

We define the operation $\set$ as follows:
\begin{align*}
    \set(\varepsilon) = \set(\row{}) &= \emptyset\\
    \set(\row{A_1, \dots, A_n}) &= \S{A_1, \dots, A_n}\\
    \set(\row{A_1, \dots, A_n|R}) &= \set(\row{A_1, \dots, A_n}) + \set(R).
\end{align*}

Note that the extension uses the sum, not the union of the two sets. This means that $\set(\row{A | \row{A}})$ should yield $\S{A, A}$ instead of $\S{A}$.

Then we get the following equality relation between effect rows $A$ and $B$:
\[ A \cong B \iff \set(A) = \set(B). \]
In typing judgments, the effect row is an overapproximation of the effects that actually used by the expression. We freely use set operations in the typing judgments, implicitly calling the $\set$ function on the operands where required. An omitted effect row is treated as an empty effect row ($\row{}$).

Any effect prefixed with a $!$ is a higher-order effect, which must elaborated instead of handled. Due to this distinction, we define the operations $H(R)$ and $A(R)$ representing the higher-order and first-order subsets of the effect rows, respectively. The same operators are applied as predicates on individual effects, so the operations on rows are defined as:
\[ 
    H(\Delta) = \S{ \phi \in \Delta \given H(\phi) }
    \qquad
    \text{and}
    \qquad
    A(\Delta) = \S{ \phi \in \Delta \given A(\phi) }.
\]

\TODO{Talk about \autocite{leijen_extensible_2005,leijen_koka_2014}.}

During type checking effect rows are represented as a pair consisting of a multiset of effects and an optional extension variable. In this section we will use a more explicit notation than the syntax of Elaine by using the multiset representation directly. Hence, a row $\row{A_1,\dots,A_n|e_A}$ is represented as the multiset $\{A_1,\dots,A_n\} + e_A$.

Like with regular Hindley-Milner type inference, two rows can be unified if we can find a substitution of effect row variables that make the rows equal. For effect rows, this yields 3 distinct cases.

If both rows are closed (i.e. have no extension variable) there are no variables to be substituted, and we just employ multiset equality. That is, to unify rows $A$ and $B$ we check that $A = B$. If that is true, we do not need to unify further and unification has succeeded. Otherwise, we cannot make any substitutions to make them equal and unification has failed.

If one of the rows is open, then the set of effects in that row need to be a subset of the effects in the other row. To unify the rows
\[ A + e_A \quad\text{and}\quad B \]
we assert that $A \subseteq B$. If that is true, we can substitute $e_n$ for the effects in $B - A$.

Finally, there is the case where both rows are open:
\[ A + e_A \quad\text{and}\quad B + e_B. \]
In this case, unification is always possible, because both rows can be extended with the effects of the other. We create a fresh effect row variable $e_C$ with the following substitutions:
\begin{align*}
    e_A &\to (B - A) + e_C \\
    e_B &\to (A - B) + e_C.
\end{align*}
In other words, $A$ is extended with the effects that are in $B$ but not in $A$ and, similarly, $B$ is extended with the effects in $A$ but not in $A$.

\TODO{Formalize this here and move some stuff to the tour chapter.}

\section{Typing judgments}

The context $\Gamma = (\Gamma_M, \Gamma_V, \Gamma_E, \Gamma_\Phi)$ consists of the following parts:
\begin{align*}
    \Gamma_M &: x \to (\Gamma_V, \Gamma_E, \Gamma_\Phi) & \text{module to context}\\
    \Gamma_V &: x \to \sigma & \text{variable to type scheme}\\
    \Gamma_T &: x \to T & \text{identifier to custom type}
    \Gamma_\Phi &: x \to \S{s_1,\dots,s_n} & \text{effect to operation signatures}
\end{align*}

\info{A $\Gamma_T$ for data types might be added.}

Whenever one of these is extended, the others are implicitly passed on too, but when declared separately, they not implicitly passed. For example, $\Gamma''$ is empty except for the single $x: T$, whereas $\Gamma'$ implicitly contains $\Gamma_M$, $\Gamma_E$ \& $\Gamma_\Phi$.
\[ \Gamma'_V = \Gamma_V, x: T \qquad \Gamma''_V = x: T \]

If the following invariants are violated there should be a type error:

\begin{itemize}
    \item The operations of all effects in scope must be disjoint.
    \item Module names are unique in every scope.
    \item Effect names are unique in every scope.
\end{itemize}

\subsection{Type inference}

We have the usual generalize and instantiate rules. But, the ``generalize'' rule requires an empty effect row.

\begin{gather*}
    \infer{
        \Gamma \vdash e : \forall \alpha. \sigma
    }{
        \Gamma \vdash e : \sigma
        \qquad
        \alpha \not\in \ftv(\Gamma)
    }
    \qquad
    \infer{
        \Gamma \vdash e : \sigma[\alpha \mapsto T']
    }{
        \Gamma \vdash e : \forall \alpha. \sigma
    }
\end{gather*}

Where $\ftv$ refers to the free type variables in the context.

\subsection{Expressions}
The typing rules for expressions are given below. The rules for handlers and elaborations are worth considering separately and are listed in \cref{sec:spechandler,sec:specelabs}.

\begin{gather*}
    \infer{
        \Gamma \vdash x : \Delta\;\tau
    }{
        \Gamma_V(x) = \Delta\;\tau
    }
    \qquad
    \infer{
        \Gamma \vdash \S{e} : \Delta\;\tau
    }{
        \Gamma \vdash e : \Delta\;\tau
    }
    \qquad
    \infer{
        \Gamma \vdash \unit: \Delta\;\unit
    }{}
    \qquad
    \infer{
        \Gamma \vdash number : \Delta\;\mcode{Int}
    }{}
    \\\\
    \qquad
    \infer{\Gamma \vdash \true : \Delta\;\Bool}{}
    \qquad
    \infer{\Gamma \vdash \false : \Delta\;\Bool}{}
    \qquad
    \infer{
        \Gamma \vdash string : \Delta\;\mcode{String}
    }{}
    \\\\
    \infer{
        \Gamma \vdash \tok{(} e_1\tok{,} \dots\tok{,} e_n\tok{)} : (T_1, \dots, T_n)
    }{
        \left[ \Gamma \vdash e_i : T_i \right]_{1\leq i\leq n}
    }
    \qquad
    \infer{
        \Gamma \vdash e\tok{(}e_1\tok{,} \dots\tok{,} e_n\tok{)}: \Delta\;\tau
    }{
        \Gamma \vdash e: (\tau_1, \dots, \tau_n) \to \Delta\;\tau
        \qquad
        \Gamma \vdash e_i : \Delta\;\tau_i
    }
    \\\\
    \infer{
        \Gamma \vdash \fn\tok{(}x_1\tok{:}\;\tau_1\tok{,}\;\dots\tok{,}\;x_n\tok{:}\;\tau_n\tok{)}\;T\; \{e\}: \Delta\;(\tau_1,\dots, \tau_n) \to T
    }{
        \Gamma_V, x_1: \row{}\tau_1, \dots, x_n: \row{}\tau_n \vdash c : T
    }
    \\\\
    \infer{
        \Gamma \vdash \cond{e_1}{e_2}{e_3} : \Delta\;\tau
    }{
        \Gamma \vdash e_1 : \Delta\;\Bool
        \qquad
        \Gamma \vdash e_2 : \Delta\;\tau
        \qquad
        \Gamma \vdash e_3 : \Delta\;\tau
    }
    \\\\
    \infer{
        \Gamma \vdash \kw{let}\;x\;\tok{=}\;e_1\tok{;}\;e_2 : \Delta\;\tau'
    }{
        \Gamma \vdash e_1 : \Delta\;\tau
        \qquad
        \Gamma_V, x: \tau \vdash e_2 : \Delta\;\tau'
    }
    \qquad
    \infer{
        \Gamma \vdash \kw{let}\;\kw{rec}\;x\;\tok{=}\;e_1\tok{;}\;e_2 : \Delta\;\tau_2
    }{
        \Gamma_V, x: \tau \vdash e_1 : \Delta\;\tau_1
        \qquad
        \Gamma_V, x: \tau \vdash e_2 : \Delta\;\tau_2
    }
    \\\\
    \infer{
        \Gamma \vdash \kw{match}\;e\;\tok{\{}\;p_1\;\kw{=>}\; e_1\tok{,}\dots\tok{,} p_n\;\kw{=>}\;e_n\;\tok{\}} : \Delta\tau
    }{
        \Gamma \vdash e: \Delta x \qquad C_1,\dots,C_n = \Gamma_T(x)\\
        \left[
            \begin{gathered}
                c_i\tok{(}x_{i,1}\tok{,} \dots\tok{,} x_{i,m_i}\tok{)} = p_i\qquad c_i(\tau_{i,1}, \dots, \tau_{i,m_i}) = C_i
                \\
                \Gamma, x_{i,1} : \tau_{i,1}, \dots, x_{i,m_i} : \tau_{i,m_i} \vdash e_i : \Delta\tau
            \end{gathered}
        \right]_{1\leq i\leq n}
    }
\end{gather*}

\subsection{Declarations and Modules}

The modules are gathered into $\Gamma_M$ and the variables that are in scope are gathered in $\Gamma_V$. The $\Rightarrow$ relation specifies the bindings that a declaration generates. The right-hand side of this relation is a public with the private and public declarations. Without modifier, the public declarations are empty, but with the \kw{pub} keyword, the private bindings are copied into the public declarations.

\begin{gather*}
    \infer{
        \Gamma \vdash d_1 \dots d_n \Rightarrow (\Gamma_n; \Gamma)
    }{
        \Gamma_{i-1} \vdash d_i \Rightarrow (\Gamma'_i; \Gamma'_{\text{pub}, i})
        \qquad
        \Gamma_i = \Gamma_{i-1}, \Gamma'_i
        \qquad
        \Gamma \vdash \Gamma'_{\text{pub},1}, \dots, \Gamma'_{\text{pub},n}
    }
    \\\\
    % module
    \infer{
        \Gamma \vdash \kw{mod}\;x\;\S{ d_1 \dots d_n } \Rightarrow (\Gamma_{M,x}; \epsilon)
    }{
        \Gamma \vdash d_1 \dots s_n \Rightarrow (\Gamma_{\text{priv}};\Gamma_{\text{pub}})
        \qquad
        \Gamma_{M,x} = x : \Gamma_{\text{pub}}
    }
    \\\\
    \qquad
    % public declaration
    \infer{
        \Gamma \vdash \kw{pub}\;d \Rightarrow (\Gamma'; \Gamma')
    }{
        \Gamma \vdash d \Rightarrow (\Gamma', \epsilon)
    }
    \qquad
    % Import
    \infer{
        \Gamma \vdash \kw{use}\;x\tok{;} \Rightarrow (\Gamma_M(x), \epsilon)
    }{}
    \\\\
    % Type declaration
    \infer{
        \Gamma \vdash \type\;x \;
        \{ x_1(\tau_{1,1}, \dots, \tau_{1,n_1}), \dots, x_m(\tau_{m,1}, \dots, \tau_{m,n_m}) \} \Rightarrow (\Gamma', \epsilon)
    }{
        f_i = \forall \alpha. (\tau_{i,1}, \dots, \tau_{i,n_i}) \to \alpha\;x
        \\
        \Gamma'_V = x_1: f_1,\dots,x_m: f_m
        \qquad
        \Gamma'_T = x: \{x_1(\tau_{1,1}, \dots, \tau_{1,n_1}), \dots, x_m(\tau_{m,1}, \dots, \tau_{m,n_m})\}
    }
    \\\\
    % Global value
    \infer{
        \Gamma \vdash \kw{let}\;x\;\tok{=}\;e\tok{;} \Rightarrow (\Gamma', \epsilon)
    }{
        \Gamma \vdash e : \tau
        \qquad
        \Gamma'_V = x : \tau
    }
    \qquad
    \infer{
        \Gamma \vdash \kw{let}\;\kw{rec}\;x\;\tok{=}\;e\tok{;} \Rightarrow (\Gamma'; \epsilon)
    }{
        \Gamma, x: \tau \vdash e : \tau
        \qquad
        \Gamma'_V = x : \tau
    }
\end{gather*}

\subsection{Algebraic Effects and Handlers}\label{ssec:spechandler}
Effects are declared with the \kw{effect} keyword. The signatures of the operations are stored in $\Gamma_\Phi$. The types of the arguments and resumption must all have no effects.

A handler must have operations of the same signatures as one of the effects in the context. The names must match up, as well as the number of arguments and the return type of the expression, given the types of the arguments and the resumption. The handler type then includes the handled effect $\phi$, an ``input'' type $\tau$, and an ``output'' type $\tau'$. In most cases, these will be at least partially generic.

The handle expression will simply add the handled effect to the effect row of the inner expression \fixme{and use the input and output type}.

\begin{gather*}
    % first-order (algebraic) effect
    \infer{
        \Gamma \vdash \kw{effect}\;x\;\S{s_1, \dots, s_n}: \Gamma'
    }{
        s_i = op_i(\tau_{i,1},\dots, \tau_{i,n_i}): \tau_i
        \qquad
        \Gamma'_\Phi(x) = \S{s_1, \dots, s_n}
    }
    \\\\
    \infer{
        \Gamma \vdash \handle\;e_h\;e_c : \Delta\;\tau'
    }{
        \Gamma \vdash e_h : \handler\;\phi\;\tau\;\tau'
        \qquad
        \Gamma \vdash e_c : \row{\phi | \Delta}\;\tau
    }
    \\\\
    \infer{
        \Gamma \vdash \handler\;\S{ \return(x) \S{ e_{\text{ret}} }, o_1, \dots, o_n }
        : \handler\;\phi\;\tau\;\tau'
    }{
        A(\phi)
        \qquad
        \Gamma_\Phi(\phi) = \S{ s_1, \dots, s_n }
        \qquad
        \Gamma, x: \tau \vdash e_{\text{ret}} : \tau' 
        \\
        \left[
            \begin{gathered}
                s_i = x_i(\tau_{i,1}, \dots, \tau_{i,m_i}) \to \tau_i
                \qquad
                o_i = x_i(x_{i,1}, \dots, x_{i,m_i})\;\S{ e_i }
                \\
                \Gamma_V, resume : (\tau_i) \to \tau', x_{i,1}: \tau_{i,1}, \dots, x_{i,i_m}: \tau_{i,i_m} 
                \vdash e_i: \tau'
            \end{gathered}
        \right]_{1\leq i\leq n}
    }
\end{gather*}

\subsection{Higher-Order Effects and Elaborations}\label{ssec:specelabs}

The declaration of higher-order effects is similar to first-order effects, but with exclamation marks after the effect name and all operations. This will help distinguish them from first-order effects.

Elaborations are of course similar to handlers, but we explicitly state the higher-order effect $x!$ they elaborate and which first-order effects $\Delta$ they elaborate into. The operations do not get a continuation, so the type checking is a bit different there. As arguments, they take the effectless types they specified along with the effect row $\Delta$. Elaborations are not added to the value context, but to a special elaboration context mapping the effect identifier to the row of effects to elaborate into.
\\
\info{Later, we could add more precise syntax for which effects need to be present in the arguments of the elaboration operations.}

The \el{elab} expression then checks that an elaboration for all higher-order effects in the inner expression are in scope and that all effects they elaborate into are handled.
\\
\info{It is not possible to elaborate only some higher-order effects. We could change the behaviour to allow this later.}

\begin{gather*}
    % higher-order effect
    \infer{
        \Gamma \vdash \kw{effect}\;x!\;\S{s_1, \dots, s_n}: \Gamma'
    }{
        s_i = op_i!(\tau_{i,1}, \dots, \tau_{i,n_i}): \tau_i
        \qquad
        \Gamma'_\Phi(x!) = \S{s_1, \dots, s_n}
    }
    \\\\
    % Elaboration
    \infer{
        \Gamma \vdash \elaboration\;x! \to \Delta\;\S{o_1, \dots, o_n} : \Gamma'
    }{
        \Gamma_\Phi(x!) = \S{s_1, \dots, s_n}
        \qquad
        \Gamma'_E(x!) = \Delta
        \\
        \left[
            \begin{gathered}
                s_i = x_i!(\tau_{i,1}, \dots, \tau_{i,m_i})\;\tau_i \qquad o_i = x_i!(x_{i,1}, \dots, x_{i,m_i}) \S{ e_i }
                \\
                \Gamma,x_{i,1}: \Delta\;\tau_{i,1},\dots,x_{i,n_i}: \Delta\;\tau_{i,n_i} \vdash 
                e_i : \Delta\;\tau_i
            \end{gathered}
        \right]_{1\leq i \leq n}
            }
    \\\\
    % Elab
    \infer{
        \Gamma \vdash \elab\;e : \Delta\;\tau
    }{
        \big[
            \Gamma_E(\phi) \subseteq \Delta
        \big]_{\phi \in H(\Delta')}
        \qquad
        \Gamma \vdash e : \Delta'\;\tau
        \qquad
        \Delta = A(\Delta')
    }
    \\\\
\end{gather*}

\section{Desugaring}
To simplify the reduction rules, we simplify the AST by desugaring some constructs. This transform is given by a fold over the syntax tree with the following operation:

\begin{align*}
    D(\fn(x_1: T_1, \dots, x_n: T_n)\;T\;\{e\}) &= \lambda x_1,\dots,x_n . e \\
    D(\kw{let}\;x = e_1;\;e_2) &= (\lambda x . e_2)(e_1)\\
    D(e_1; e_2) &= (\lambda \_ . e_2)(e_1)\\
    D(\S{e}) &= e\\
\end{align*}

\section{Semantics}\label{sec:semantics}

The semantics of Elaine are defined as reduction semantics.

We use two separate contexts to evaluate expressions. The $E$ context is for all constructs except effect operations, such as \code{if}, \code{let}, and function applications. The $X_{op}$ context is the context in which a handler can reduce an operation $op$.

\begin{align*}
    E
        \IS & [] \OR E(e_1,\dots, e_n) \OR v(v_1,\dots,v_n,E,e_1,\dots,e_m) \\
        \OR & \cond{E}{e}{e} \\
        \OR & \kw{let}\;x = E;\;e \OR E;\;e\\
        \OR & \kw{handle}[E]\;e \OR \kw{handle}[v]\;E \\
        \OR & \kw{elab}[E]\;e \OR \kw{elab}[v]\;E\\
    \\
    X_{op}
        \IS & [] \OR X_{op}(e_1, \dots, e_n) \OR v(v_1, \dots, v_n, X_{op}, e_1, \dots, e_m) \\
        \OR & \cond{X_{op}}{e_1}{e_2} \\
        \OR & \kw{let}\;x = X_{op};\;e \OR X_{op};\;e \\
        \OR & \kw{handle}[X_{op}]\;e \OR \kw{handle}[h]\;{X_{op}} \text{ if } op\not\in h \\
        \OR & \elab[X_{op}]\;e \OR \kw{elab}[\epsilon]\;X_{op} \text{ if } op! \not\in e
\end{align*}

\TODO{Add some explanation}

\newcommand{\reduce}{\quad\longrightarrow\quad}
\begin{align*}
    c(v_1, \dots, v_n) \reduce& \delta(c, v_1, \dots, v_n) \\
    &\qquad\text{if } \delta(c, v_1, \dots, v_n) \text{ defined}\\
    (\lambda x_1, \dots, x_n . e) (v_1, \dots, v_n) \reduce& e[x_1 \mapsto v_1, \dots, x_n \mapsto v_n] \\
    \cond{\true}{e_1}{e_2} \reduce& e_1 \\
    \cond{\false}{e_1}{e_2} \reduce& e_2 \\
    \\
    \kw{handle}[h]\;v \reduce& e[x\mapsto v] \\
    &\qquad\text{where } \return(x) \S{ e } \in H\\
    \kw{handle}[h]\;X_{op}[op(v_1, \dots, v_n)] \reduce& e[x_1\mapsto v_1, \dots, x_n\mapsto v_n, resume \mapsto k] \\
    &\qquad \text{where } \begin{aligned}[t]
        & op(x_1, \dots, x_n) \S{e} \in h\\
        & k = \lam{y}{\kw{handle}[h]\;
        X_{op}[y]}
    \end{aligned}\\
    \kw{elab}[\epsilon]\;v \reduce& v\\
    \kw{elab}[\epsilon]\;X_{op!}[op!(e_1, \dots, e_n)] \reduce& \kw{elab}[\epsilon]\;X_{op!}[e[x_1 \mapsto e_1, \dots, x_n \mapsto e_n]] \\
    &\qquad \text{where } op!(x_1, \dots, x_n) \S{e} \in \epsilon \\
\end{align*}

\section{Standard Library}\label{sec:std}

Elaine does not include any operators. This choice was made to simplify parsing of the language. For the lack of operators, any manipulation of primitives needs to be done via the standard library of built-in functions.

These functions reside in the \el{std} module, which can be imported like any other module with the \el{use} statement to bring its contents into scope.

The full list of functions available in the \el{std} module, along with their signatures and descriptions, is given in \cref{fig:std}.

\begin{figure}[htbp]
\begin{tabular}{lllll}
& Name & Type signature &  & Description \\
\hline
Arithmetic
& \code{add}    & \code{fn(Int, Int)} & \code{Int} & addition \\
& \code{sub}    & \code{fn(Int, Int)} & \code{Int} & subtraction \\
& \code{neg}    & \code{fn(Int)}      & \code{Int} & negation \\
& \code{mul}    & \code{fn(Int, Int)} & \code{Int} & multiplication \\
& \code{div}    & \code{fn(Int, Int)} & \code{Int} & division \\
& \code{modulo} & \code{fn(Int, Int)} & \code{Int} & modulo \\
& \code{pow}    & \code{fn(Int, Int)} & \code{Int} & exponentiation \\
\hline
Comparisons
& \code{eq}  & \code{fn(Int, Int)} & \code{Bool} & equality \\
& \code{neq} & \code{fn(Int, Int)} & \code{Bool} & inequality \\
& \code{gt}  & \code{fn(Int, Int)} & \code{Bool} & greater than \\
& \code{geq} & \code{fn(Int, Int)} & \code{Bool} & greater than or equal \\
& \code{lt}  & \code{fn(Int, Int)} & \code{Bool} & less than \\
& \code{leq} & \code{fn(Int, Int)} & \code{Bool} & less than or equal \\
\hline
Boolean operations
& \code{not} & \code{fn(Bool)} & \code{Bool} & boolean negation \\
& \code{and} & \code{fn(Bool, Bool)} & \code{Bool} & boolean and \\
& \code{or} & \code{fn(Bool, Bool)} & \code{Bool} & boolean or \\
\hline
String operations
& \code{concat} & \code{fn(Bool, Bool)} & \code{Bool} & string concatenation \\
\hline
Conversions
& \code{show_int} & \code{fn(Int)} & \code{String} & integer to string \\
& \code{show_bool} & \code{fn(Bool)} & \code{String} & integer to string \\
\end{tabular}
\caption{Overview of the functions in the \el{std} module in Elaine.}
\label{fig:std}
\end{figure}

\TODO{Add all the other things I added to the standard library here too}
