\chapter{Elaine Specification}\label{chap:spec}

This chapter contains the specification for Elaine. This specification includes the syntax definition, type inference rules, reduction semantics, and the functions provided by the standard library.

\section{Syntax Definition}\label{sec:syntax}

The Elaine syntax was designed to be easy to parse. The grammar is not white-space sensitive and most constructs are unambiguously identified with keywords at the start.

The full syntax definition is given in \cref{fig:syntax} as a context-free grammar. In this grammar, $x$ represents an identifier. We use the following notation for the grammar:
\begin{itemize}
    \item a sort is declared with $::=$,
    \item the alternatives of a sort are separated by $|$,
    \item tokens are written in \tok{monospace font},
    \item $\opt{p}$ indicates that the sort $p$ is optional,
    \item $\rep{p}$ indicates that the sort $p$ can be repeated zero or more times, and
    \item $\csep{p}$ indicates that the sort $p$ can be repeated zero or more times, separated by commas.
\end{itemize}

The sorts $x$, $i$ and $s$ denote identifiers, integers and strings respectively. Identifiers must match the regex expression \code{[a-zA-Z_][a-zA-Z_0-9]*'*} and cannot be a keyword. Integers consist of numeric digits. Strings are delimited by double quotes.

\begin{figure}[p]
\begin{align*}
    \text{program}\ p
        \IS & d \dots d\\
    \\
    \text{declaration}\ d
        \IS & \opt{\tok{pub}}\ \tok{mod}\ x\ \tok{\{} \rep{d} \tok{\}}\\
        \OR & \opt{\tok{pub}}\ \tok{use}\ x\tok{;}\\
        \OR & \opt{\tok{pub}}\ \tok{let}\ \opt{\tok{rec}}\ p\ \tok{=}\ e\tok{;}\\
        \OR & \opt{\tok{pub}}\ \tok{effect}\ \phi\ \tok{\{} \csep{s} \tok{\}}\\
        \OR & \opt{\tok{pub}}\ \tok{type}\ x\ \tok{\{} \csep{c} \tok{\}}\\
    \\
    \text{block}\ b
        \IS & \tok{\{}\ es\ \tok{\}}\\
    \text{expression list}\ es
        \IS & e\\
        \OR & e\tok{;}\ es\\
        \OR & \tok{let}\ \opt{\tok{rec}}\ p\ \tok{=}\ e\tok{;}\ es\\
    \text{expression}\ e
        \IS & x\\
        \OR & \tok{()} \OR \tok{true} \OR \tok{false} \OR i \OR s \\
        \OR & \tok{(} \csep{e} \tok{)} \\
        \OR & \tok{fn} \tok{(} \csep{p} \kw{)}\ \opt{T}\ b \\
        \OR & \tok{if}\ e\ b\ \tok{else}\ b\\
        \OR & e\tok{(} \csep{e} \tok{)} \OR \phi\tok{(} \csep{e} \tok{)} \\
        \OR & \tok{handler}\ \tok{\{} \csep{o} \tok{\}}\\
        \OR & \tok{handler}\ \tok{\{} \tok{return} \tok{(}x\tok{)}\ b \tok{,}\ \csep{o} \tok{\}}\\
        \OR & \tok{handle}\tok{[}e\tok{]}\ e \\
        \OR & \tok{elaboration}\ x\tok{!}\ \tok{->}\ R\ \tok{\{} \csep{o} \tok{\}}\\
        \OR & \tok{elab}\tok{[} e \tok{]}\ e \OR \tok{elab}\ e \\
        \OR & b\\
    \\
    \text{annotatable variable}\ p
        \IS & x\ \tok{:}\ T \OR x\\
    \text{signature}\ s
        \IS & x \tok{(} \csep{T} \tok{)}\ T\\
    \text{effect clause}\ o
        \IS & x \tok{(} \csep{x} \tok{)}\ b\\
    \text{constructor}\ c
        \IS & x \tok{(} \csep{T} \tok{)}\\
    \\
    \text{type}\ T
        \IS & R\ \tau \OR \tau\\
    \text{value type}\ \tau
        \IS & x \\ 
        \OR & \tok{()} \OR \tok{Bool} \OR \tok{Int} \OR \tok{String}\\
        \OR & \tok{fn} \tok{(} \csep{T} \tok{)}\ T \\
        \OR & \tok{Handler}\tok{[}x\tok{,}\tau\tok{,}\tau\tok{]} \\
        \OR & \tok{Elab}\tok{[}x!\tok{,}R\tok{]} \\
        \OR & x\tok{[}\csep{\tau}\tok{]} \\
    \text{effect row sugar}\ R
        \IS & \tok{<} \csep{\phi} \tok{>}
        \OR \tok{<} \csep{\phi} \tok{|} R \tok{>}
        \OR \Delta \\
    \text{effect row}\ \Delta
        \IS & \tok{<>} \OR \tok{<}\phi\tok{|}\Delta\tok{>}\\
    \text{effect}\ \phi \IS & x \OR x\tok{!}
\end{align*}
\caption{Syntax definition of Elaine}
\label{fig:syntax}
\end{figure}

\section{Effect Rows}\label{sec:effectrows}

The syntax definition in \cref{fig:syntax} describes a layer of syntactic sugar for effect rows. The desugared syntax for effect rows is given as follows:
\[
\Delta \IS \tok{<>} \OR \tok{<}\phi\tok{|}\Delta\tok{>}
\]
We transform the syntactic sugar form into the standard form with the $\Norm$ operation defined in \cref{fig:normalization}, which maps $R$ to $\Delta$. This operation is applied to all occurrences of $R$ in the syntax tree. Additionally, our typing judgments require operations that split an effect row into algebraic and higher-order effects, denoted $\A$ and $\Ho$, respectively. The definition of these operations is given in \cref{fig:HandA}.

Effect rows should be considered equivalent up to reordering. Following \textcite{leijen_koka_2014}, we define the $\equiv$ relation to denote effect row equivalence in \cref{fig:rowequiv}. This equivalence is implicitly used whenever effect rows are asserted to be equivalent in the typing judgments.

\begin{figure}[htbp]
\begin{subfigure}{\textwidth}
    \begin{align*}
        \Norm(\tok{<>}) &= \tok{<>} \\
        \Norm(\tok{<}\tok{|} R \tok{>}) &= \Norm(R) \\
        \Norm(\tok{<} \phi_1 \tok{>}) &= \tok{<} \phi_1 \tok{|} \tok{<>} \tok{>} \\
        \Norm(\tok{<}\phi_1\tok{,}\dots\tok{,}\phi_n\tok{>})
        &= \tok{<} \phi_1 \tok{|} \Norm(\tok{<} \phi_2\tok{,}\dots\tok{,}\phi_n \tok{>}) \tok{>}\\ 
        \Norm(\tok{<}\phi_1\tok{,}\dots\tok{,}\phi_n\tok{|} R \tok{>})
        &= \tok{<} \phi_1 \tok{|}\Norm(\tok{<} \phi_2\tok{,}\dots\tok{,}\phi_n \tok{|} R \tok{>}) \tok{>}
    \end{align*}
    \caption{Definition of normalization $\Norm$ from the syntactic sugar for effect rows $R$ to an effect row $\Delta$.}
    \label{fig:normalization}
\end{subfigure}
\begin{subfigure}{\textwidth}
    \begin{align*}
        \Ho(\tok{<>}) &= \tok{<>} & \A(\tok{<>}) &= \tok{<>} \\
        \Ho(\tok{<} x \tok{|} \Delta \tok{>}) &= \Ho(\Delta) &
        \A(\tok{<} x \tok{|} \Delta \tok{>}) &= \tok{<} x \tok{|} \A(\Delta) \tok{>} \\
        \Ho(\tok{<} x! \tok{|} \Delta \tok{>}) &= \tok{<} x! \tok{|} \Ho(\Delta) \tok{>} &
        \A(\tok{<} x! \tok{|} \Delta \tok{>}) &= \A(\Delta)
    \end{align*}
    \caption{Definition of the $\Ho$ and $\A$ operations, which reduce an effect row to only higher-order or algebraic effects, respectively.}
    \label{fig:HandA}
\end{subfigure}
\begin{subfigure}{\textwidth}
    \begin{mathpar}
    \hfill\boxed{\Delta_1 \equiv \Delta_2}\\
    \inferrule[Eq-Refl]{ }{ \Delta \equiv \Delta}
    \and
    \inferrule[Eq-Head]{
        \Delta_1 \equiv \Delta_2
    }{
        \tok{<} \phi \tok{|} \Delta_1 \tok{>}
        \equiv
        \tok{<} \phi \tok{|} \Delta_2 \tok{>}
    }
    \\\\
    \inferrule[Eq-Trans]{
        \Delta_1 \equiv \Delta_2
        \\
        \Delta_2 \equiv \Delta_3
    }{
        \Delta_1 \equiv \Delta_3
    }
    \and
    \inferrule[Eq-Swap]{
        \phi_1 \not\equiv \phi_2
    }{
        \tok{<} \phi_1 \tok{|} \tok{<} \phi_2 \tok{|} \Delta \tok{>} \tok{>}
        \\
        \tok{<} \phi_2 \tok{|} \tok{<} \phi_1 \tok{|} \Delta \tok{>} \tok{>}
    }
    \end{mathpar}
    \caption{Effect row equivalence}
    \label{fig:rowequiv}
\end{subfigure}
\caption{Operations and relations on effect rows.}
\end{figure}

\section{Type System}

We give a declarative specification of the type system of Elaine. This specification consists of two parts: inference rules for expressions and inference rules for declarations.

The context $\Gamma = (\Gamma_M; \Gamma_V; \Gamma_E; \Gamma_\Phi)$ consists of the following parts:
\begin{align*}
    \Gamma_M &: x \to \Gamma&& \text{module to context}\\
    \Gamma_V &: x \to \sigma && \text{variable to type scheme}\\
    \Gamma_T &: x \to T && \text{identifier to custom type}\\
    \Gamma_\Phi &: x \to \S{s_1,\dots,s_n} && \text{effect to operation signatures}
\end{align*}
In the typing judgments, we often need to extend just one of these sub-contexts. Therefore, if we extend one, the rest is implicitly passed to. For example, the following expressions are equivalent:
\begin{gather*}
    \Gamma'_V = \Gamma_V, x: T \\
    \Gamma' = (\Gamma_M\ ;\ \Gamma_V, x: T\ ;\ \Gamma_E\ ;\ \Gamma_\Phi)
\end{gather*}

In the typing judgments below, we assume that all types are explicitly specified for function arguments and return types. Hindley-Milner type inference can be used to infer the types if missing. This is also done in the Elaine prototype. In addition to the typing judgments, we assert that all effects, effect operations and modules have unique names within any scope.

The typing judgments for expressions inductively define a ternary relation
\[ \Gamma \vdash e : \Delta\tau, \]
where $\Gamma$ ranges over contexts, $e$ ranges over expressions and $\Delta\;\tau$ ranges over pair of effect rows and value types. This relation should be read as: ``in the context $\Gamma$, the expression $e$ has type $\Delta\tau$''. This relation is fairly standard, apart from the inclusion of the effect row. Note that in this specification, the effect row is an overapproximation of the effect which are present. This means, for example, that the (pure) boolean value \el{true} has the type $\Delta\Bool$ for every effect row $\Delta$, as rule \textsc{E-True} shows. A common pattern in the judgments is that many expressions that are evaluated in sequence all have the same effect row. This can be seen clearly in the \textsc{E-If} and \textsc{E-Tuple} rules. A function application (\textsc{E-App}) has an effect row that matches the same effect row on the return type of the called function. 

Handlers have the type \lstinline[mathescape]{Handler[$\phi$,$\tau$,$\tau'$]}. In this type, the $\phi$ variable represents the effect that it handles, $\tau$ is the type of the sub-computation and $\tau'$ is the type of the \el{handle} that handles a computation with type $\tau$. The $\tau$ and $\tau'$ types may be related. For example, a handler for the abort effect can be represented by the following type scheme:
\[ \forall \alpha.\ \kw{Handler[Abort,}\alpha\kw{,Maybe[}\alpha\kw{]]}. \]
If no \el{return} branch is specified, it is assumed to be the identity function and hence $\tau$ is equal to $\tau'$ in that case, as written in rule \textsc{E-Handler}.

Elaborations have a similar type to handlers: \lstinline[mathescape]{Elab[$\phi$,$\Delta$]}. Here, $\phi$ again represents the effect this elaboration is for. The effect row $\Delta$ is the set of algebraic effects it elaborates into. Therefore, \textsc{E-Elaboration} asserts that $\Delta$ is algebraic. When an elaboration is applied (\textsc{E-Elab}), this effect row must be unifiable with the effect row for the return type of of \el{elab}.

Rule \textsc{E-ImplicitElab} is different from \textsc{E-Elab}, because it elaborates all higher-order effects. Therefore, it requires that exactly one elaboration is in scope for each higher-order effect. In this rule, $A(\Delta')$ represents the algebraic subset of the effects in $\Delta'$. The uniqueness is important here to keep the implicit elaboration resolution predictable as explained in \cref{chap:elabres}.

The inference rules for declarations define a different relation
\[ \Gamma \vdash d \Rightarrow (\Gamma_{\text{priv}}, \Gamma_{\text{pub}}), \]
where $\Gamma$ again ranges over contexts and $d$ ranges over declarations and sequences of declarations. The pair of contexts represent the private and public context that $d$ generates. The relation should be read as ``in the context $\Gamma$, the declaration $d$ generates the bindings in $\Gamma_{\text{priv}}$ and exposes the bindings in $\Gamma_{\text{pub}}$.'' Both contexts only contain new bindings; they are not additive. The private context is always a subset of the public context.

All declarations generate only private bindings with an empty context for public bindings, written $\varepsilon$. The \textsc{D-Pub} rule then ensures that this private context is duplicated to the public context. Therefore, if the declarations in a module generate $(\Gamma_{\text{priv}};\Gamma_{\text{pub}})$, then only $\Gamma_{\text{pub}}$ should be stored in $\Gamma_M$, which is specified by the \textsc{D-Mod} rule.

The \textsc{D-Type} rule ensures two things: it adds the type to the type context $\Gamma_T$ and adds all constructors as functions to the variable context $\Gamma_V$. Because all constructors are modelled as functions, we do not need special rules for constructors in the expression inference rules. The \textsc{D-AlgebraicEffect} and \textsc{D-HigherOrderEffect} rules are split to ensure that all higher-order operations have an \el{!} suffix. Apart from that difference, these rules are identical.

\setlength{\fboxsep}{2\fboxsep}

\begin{figure}[p]
\vspace*{-2em}
\begin{mathpar}
    \hfill\boxed{\Gamma \vdash e : T}\\
    \inferrule[E-Gen]{
        \Gamma \vdash e : \sigma
        \qquad
        \alpha \not\in \ftv(\Gamma)
    }{
        \Gamma \vdash e : \forall \alpha. \sigma
    }
    \and
    \inferrule[E-Inst]{
        \Gamma \vdash e : \forall \alpha. \sigma
        }{
        \Gamma \vdash e : \sigma[\alpha \mapsto T']
    }
    \and
    \inferrule[E-Var]{
        \Gamma_V(x) = \Delta\;\tau
        }{
        \Gamma \vdash x : \Delta\;\tau
    }
    \and
    \inferrule[E-Block]{
        \Gamma \vdash es : \Delta\;\tau
    }{
        \Gamma \vdash \tok{\{}es\tok{\}} : \Delta\;\tau
    }
    \and
    \inferrule[E-Unit]{ }{
        \Gamma \vdash \unit: \Delta\;\unit
    }
    \and
    \inferrule[E-Int]{ }{
        \Gamma \vdash i : \Delta\;\kw{Int}
    }
    \and
    \inferrule[E-True]{ }{\Gamma \vdash \true : \Delta\;\Bool}
    \and
    \inferrule[E-False]{ }{\Gamma \vdash \false : \Delta\;\Bool}
    \and
    \inferrule[E-String]{ }{
        \Gamma \vdash s : \Delta\;\kw{String}
    }
    \and
    \inferrule[E-Tuple]{
        \left[ \Gamma \vdash e_i : \Delta \tau_i \right]_{1\leq i\leq n}
        }{
        \Gamma \vdash \tok{(} e_1\tok{,} \dots\tok{,} e_n\tok{)} : \Delta\;(\tau_1, \dots, \tau_n)
    }
    \and
    \inferrule[E-Seq]{
        \Gamma \vdash e : \Delta\;\tau
        \\
        \Gamma \vdash es : \Delta\;\tau'
    }{
        \Gamma \vdash e; es : \Delta\;\tau'
    }
    \and
    \inferrule[E-Let]{
        \Gamma \vdash e : \Delta\;\tau
        \\
        \Gamma_V, x: \tau \vdash es : \Delta\;\tau'
        }{
        \Gamma \vdash \kw{let}\;x\;\tok{=}\;e\tok{;}\;es : \Delta\;\tau'
    }
    \and
    \inferrule[E-LetRec]{
        \Gamma_V, x: \tau \vdash e : \Delta\;\tau_1
        \\
        \Gamma_V, x: \tau \vdash es : \Delta\;\tau_2
    }{
        \Gamma \vdash \kw{let}\;\kw{rec}\;x\;\tok{=}\;e\tok{;}\;es : \Delta\;\tau_2
    }

    \and
    \inferrule[E-FuncDef]{
        \Gamma_V, x_1: \tok{<>}\tau_1, \dots, x_n: \tok{<>}\tau_n \vdash b : T
    }{
        \Gamma \vdash \fn\tok{(}x_1\tok{:}\;\tau_1\tok{,}\;\dots\tok{,}\;x_n\tok{:}\;\tau_n\tok{)}\;T\; b: \Delta\;(\tau_1,\dots, \tau_n) \to T
    }
    \and
    \inferrule[E-App]{
        \Gamma \vdash e: (\tau_1, \dots, \tau_n) \to \Delta\;\tau
        \\\\
        [\Gamma \vdash e_i : \Delta\;\tau_i]_{1\leq i\leq n}
    }{
        \Gamma \vdash e\tok{(}e_1\tok{,} \dots\tok{,} e_n\tok{)}: \Delta\;\tau
    }
    \and
    \inferrule[E-If]{
        \Gamma \vdash e : \Delta\;\Bool
        \\\\
        \Gamma \vdash b_1 : \Delta\;\tau
        \\\\
        \Gamma \vdash b_2 : \Delta\;\tau
    }{
        \Gamma \vdash \kw{if}\ e\ b_1\ \kw{else}\ b_2 : \Delta\;\tau
    }
    \and
    \inferrule[E-Match] {
        \Gamma \vdash e: \Delta\;\tau \\ \S{c_1,\dots,c_n} = \Gamma_T(\tau) \\\\
        \left[
            { \begin{gathered}
                x_i\tok{(}x_{i,1}\tok{,} \dots\tok{,} x_{i,m_i}\tok{)} = p_i\qquad x_i(\tau_{i,1}, \dots, \tau_{i,m_i}) = c_i
                \\
                \Gamma, x_{i,1} : \tau_{i,1}, \dots, x_{i,m_i} : \tau_{i,m_i} \vdash e_i : \Delta\tau'
            \end{gathered} }
        \right]_{1\leq i\leq n}
    }{
        \Gamma \vdash \kw{match}\;e\;\tok{\{}\;p_1\;\kw{=>}\; e_1\tok{,}\dots\tok{,} p_n\;\kw{=>}\;e_n\;\tok{\}} : \Delta\tau'
    }
    \and
    % Elab
    \inferrule[E-Elab]{
        \Gamma \vdash e_E : \Delta\;\kw{Elab}[x!, \Delta]
        \\\\
        \Gamma \vdash e_c : \tok{<} x! \tok{|} \Delta \tok{>}\;\tau
    }{
        \Gamma \vdash \elab\tok{[}e_E\tok{]}\;e_c: \Delta\tau
    }
    \and
    % Implicit Elab
    \inferrule[E-ImplicitElab]{
        \big[
            \exists!\ x.\ \Gamma_V(x) = \kw{Elab}[\phi,\Delta]
        \big]_{\phi \in \Ho(\Delta')}
        \\\\
        \Gamma \vdash e : \Delta'\;\tau
        \\
        \Delta = \A(\Delta')
    }{
        \Gamma \vdash \elab\;e : \Delta\;\tau
    }
    \and
    \inferrule[E-Handle]{
        \Gamma \vdash e_h : \Delta\;\kw{Handler}[\phi,\tau,\tau']
        \\\\
        \Gamma \vdash e_c : \tok{<}\phi \tok{|} \Delta \tok{>}\;\tau
    }{
        \Gamma \vdash \handle\tok{[}e_h\tok{]}\;e_c : \Delta\;\tau'
    }
    \and
    \inferrule[E-HandlerNoRet]{
        \Gamma \vdash \handler\;\tok{\{} \return\tok{(}x\tok{)} \tok{\{} x \tok{\}}\tok{,} o_1\tok{,} \dots\tok{,} o_n \tok{\}}
        : \kw{Handler}[\phi,\tau,\tau]
    }{
        \Gamma \vdash \handler\;\tok{\{} o_1\tok{,} \dots\tok{,} o_n \tok{\}}
        : \kw{Handler}[\phi,\tau,\tau]
    }
    \and
    \inferrule[E-Handler]{
        \Gamma_\Phi(\phi) = \S{s_1, \dots, s_n }
        \\
        \Gamma, x: \tau \vdash e_{\text{ret}} : \tau' 
        \\\\
        \left[
            { \begin{gathered}
                s_i = x_i\tok{(}\tau_{i,1}\tok{,} \dots\tok{,} \tau_{i,m_i}\tok{)} \to \tau_i
                \qquad
                o_i = x_i\tok{(}x_{i,1}\tok{,} \dots\tok{,} x_{i,m_i}\tok{)}\;\tok{\{} e_i \tok{\}}
                \\
                \Gamma_V, \kw{resume}\;: (\tau_i) \to \tau', x_{i,1}: \tau_{i,1}, \dots, x_{i,m_i}: \tau_{i,m_i}
                \vdash e_i: \tau'
            \end{gathered} }
        \right]_{1\leq i\leq n}
    }{
        \Gamma \vdash \handler\;\tok{\{} \return\tok{(}x\tok{)} \tok{\{} e_{\text{ret}} \tok{\}}\tok{,} o_1\tok{,} \dots\tok{,} o_n \tok{\}}
        : \kw{Handler}[\phi,\tau,\tau']
    }
    \and
    % Elaboration
    \inferrule[E-Elaboration]{
        \Gamma_\Phi(x!) = \S{s_1, \dots, s_n}
        \\
        \Delta = A(\Delta)
        \\\\
        \left[
            { \begin{gathered}
                s_i = x_i\tok{!}\tok{(}\tau_{i,1}\tok{,} \dots\tok{,} \tau_{i,m_i}\tok{)}\;\tau_i \qquad o_i = x_i!\tok{(}x_{i,1}\tok{,} \dots\tok{,} x_{i,m_i}\tok{)} \tok{\{} e_i \tok{\}}
                \\
                \Gamma,x_{i,1}: \Delta\;\tau_{i,1},\dots,x_{i,m_i}: \Delta\;\tau_{i,m_i} \vdash 
                e_i : \Delta\;\tau_i
            \end{gathered} }
        \right]_{1\leq i \leq n}
    }{
        \Gamma \vdash \elaboration\;x! \to \Delta\;\S{o_1, \dots, o_n} : \kw{Elab}[x!, \Delta]
    }
\end{mathpar}
\caption{Inference rules for expressions.}
\end{figure}

\begin{figure}[p]
\begin{mathpar}
    \hfill\boxed{\Gamma \vdash d \Rightarrow (\Gamma_{\text{priv}};\Gamma_{\text{pub}})}\\
    \inferrule[D-Seq]{
        \Gamma \vdash d_1 \Rightarrow (\Gamma_{\text{priv},1},\Gamma_{\text{pub,1}})
        \\
        \Gamma,\Gamma_{\text{priv},1} \vdash d_2\dots d_n \Rightarrow (\Gamma_{\text{priv},n},\Gamma_{\text{pub,n}})
    }{
        \Gamma \vdash d_1 \dots d_n \Rightarrow (\Gamma_{\text{priv},1},\Gamma_{\text{priv},n};\ \Gamma_{\text{pub},1},\Gamma_{\text{pub},n})
    }
    \and
    % module
    \inferrule[D-Mod]{
        \Gamma \vdash d_1 \dots d_n \Rightarrow (\Gamma_{\text{priv}};\Gamma_{\text{pub}})
        \\
        \Gamma_{M,x} = x : \Gamma_{\text{pub}}
    }{
        \Gamma \vdash \kw{mod}\;x\;\tok{\{} d_1 \dots d_n \tok{\}} \Rightarrow (\Gamma_{M,x}; \varepsilon)
    }
    \and
    % public declaration
    \inferrule[D-Pub]{
        \Gamma \vdash d \Rightarrow (\Gamma'; \varepsilon)
    }{
        \Gamma \vdash \kw{pub}\;d \Rightarrow (\Gamma'; \Gamma')
    }
    \and
    % Import
    \inferrule[D-Use]{ }{
        \Gamma \vdash \kw{use}\;x\tok{;} \Rightarrow (\Gamma_M(x); \varepsilon)
    }
    \and
    % Global value
    \inferrule[D-Let]{
        \Gamma \vdash e : \tau
        \qquad
        \Gamma'_V = x : \tau
    }{
        \Gamma \vdash \kw{let}\;x\;\tok{=}\;e\tok{;} \Rightarrow (\Gamma'; \varepsilon)
    }
    \and
    \inferrule[D-LetRec]{
        \Gamma, x: \tau \vdash e : \tau
        \qquad
        \Gamma'_V = x : \tau
    }{
        \Gamma \vdash \kw{let}\;\kw{rec}\;x\;\tok{=}\;e\tok{;} \Rightarrow (\Gamma'; \varepsilon)
    }
    \and
    % Type declaration
    \inferrule[D-Type]{
        f_i = \forall \alpha. (\tau_{i,1}, \dots, \tau_{i,n_i}) \to \alpha\;x
        \\
        \Gamma'_V = x_1: f_1,\dots,x_m: f_m
        \qquad
        \Gamma'_T = x: \{x_1(\tau_{1,1}, \dots, \tau_{1,n_1}), \dots, x_m(\tau_{m,1}, \dots, \tau_{m,n_m})\}
    }{
        \Gamma \vdash \type\;x \;
        \tok{\{} x_1\tok{(}\tau_{1,1}\tok{,} \dots\tok{,} \tau_{1,n_1}\tok{)}\tok{,} \dots\tok{,} x_m\tok{(}\tau_{m,1}\tok{,} \dots\tok{,} \tau_{m,n_m}\tok{)} \tok{\}} \Rightarrow (\Gamma'; \varepsilon)
    }
    
    \and
    % algebraic effect
    \inferrule[D-AlgebraicEffect]{
        s_i = op_i\tok{(}\tau_{i,1}\tok{,}\dots\tok{,} \tau_{i,n_i}\tok{)}: \tau_i
        \qquad
        \Gamma'_\Phi(x) = \tok{\{}s_1\tok{,} \dots\tok{,} s_n\tok{\}}
    }{
        \Gamma \vdash \kw{effect}\;x\;\tok{\{}s_1\tok{,} \dots\tok{,} s_n\tok{\}}: (\Gamma'; \varepsilon)
    }
    \and
    % higher-order effect
    \inferrule[D-HigherOrderEffect]{
        s_i = op_i\tok{!}\tok{(}\tau_{i,1}\tok{,} \dots\tok{,} \tau_{i,n_i}\tok{)}: \tau_i
        \qquad
        \Gamma'_\Phi(x!) = \S{s_1, \dots, s_n}
    }{
        \Gamma \vdash \kw{effect}\;x\tok{!}\;\tok{\{}s_1\tok{,} \dots\tok{,} s_n\tok{\}}: (\Gamma'; \varepsilon)
    }
\end{mathpar}
\caption{Inference rules for declarations.}
\end{figure}

\section{Desugaring}
To simplify the reduction rules, we simplify the AST by desugaring some constructs. This transform is given by a fold over the syntax tree with the operation $D$ defined below. If a construct does not match one of the cases below, it is not transformed.

\begin{align*}
    D(\fn(x_1: T_1, \dots, x_n: T_n)\;T\;\{e\}) &= \lambda x_1,\dots,x_n . e \\
    D(\kw{let}\;x = e_1;\;e_2) &= (\lambda x . e_2)(e_1)\\
    D(\kw{let}\;\kw{rec}\;x = e_1;\;e_2) &= (\lambda x . e_2)((\lambda x . x(x))(\lambda x. e_1[x\mapsto x(x)]))\\
    D(e_1; e_2) &= (\lambda \_ . e_2)(e_1)\\
    D(\S{e}) &= e\\
\end{align*}

\section{Semantics}\label{sec:semantics}
\newcommand{\reduce}{\quad\longrightarrow\quad}

We give two descriptions of the semantics of Elaine. First, we give an informal description of the denotational semantics to highlight the connection between Elaine and hefty algebras. There is no full specification for the denotational semantics of Elaine. Second, we give a formal small-step reduction semantics for the evaluation of expressions.

\subsection{Denotational Semantics}\label{ssec:denosem}
\newcommand\BB[1]{\left\llbracket\mcode{#1}\right\rrbracket}
\newcommand\bb[1]{\left\llbracket\mcode{#1}\right\rrbracket}

In this section, we give an intuition for the intended denotational semantics of Elaine. The $\llbracket\cdot\rrbracket$ operator is used to represent the denotational semantics.

In languages based on algebraic effects, such as Koka, the semantics of effect rows are free monad. So, the denotational semantics of Koka's effect row notation would then be given as follows:
\[
  \BB{() -> e a}
  \qquad=\qquad
  \mcode{() -> Free}\ \BB{e}\ \BB{a}.
\]
However, the semantics for Elaine's effect rows are hefty trees instead of free monads. Therefore, effects correspond to higher-order functors of a hefty tree and algebraic effects need to be lifted. In general, this can be expressed as:
\[
  \BB{fn() -> e a}
  \qquad=\qquad
  \mcode{() -> Hefty (Lift}\ \bb{A(e)}\ \mcode{+}\ \bb{H(e)}\mcode{)}\ \BB{a}.
\]
We conjecture that the denotational semantics of Elaine expressions are hefty trees. However, the semantics of Elaine are still unclear, because the \el{elab} construct does not correspond directly to hefty algebras. To see why, recall that elaboration in hefty algebras returns a free monad. To elaborate multiple higher-order effects, the elaborations of the effects need to be composed using \hs{^} operator.
%
Elaine's \el{elab} construct, however, returns a type with only the elaborated effect removed and allows other higher-order effects to be left in the effect row.
%
Composition of elaborations is then no longer necessary, because individual elaborations can be applied in sequence. For example, to elaborate a computation that uses both the catch and reader effect, Elaine allows the following expression:

\begin{lst}{Elaine}
let main = elab[eCatch] elab[eReader] { ... };
\end{lst}
%
This provides more flexibility than composition, since elaborations can then be applied to specific sub-computations of elaborations. This means that Elaine is more flexible than hefty algebras. For example, the following example does not have a clear semantics as hefty tree:

\begin{lst}{Elaine}
# Note: valid in Elaine, but not in hefty algebras!
let main = elab[eReader] {
    elab[eCatch1] a();
    elab[eCatch2] b()
};
\end{lst}

Therefore, the semantics for Elaine's \el{elab} construct are left to future work.

\subsection{Small-step Reduction Semantics}
The reduction semantics for Elaine are given in \cref{fig:reductionsemantics}. It is given in the form of two contexts $E$ and $X_{op}$ and a reduction relation $\longrightarrow$.

The $E$ context is used for all reduction rules except effect operations, such as \code{if}, \code{let}, and function applications. The $X_{op}$ context is the context in which a handler can reduce an operation $op$. The two contexts are mostly equivalent, except for the fact that an $X_{op}$ cannot enter the sub-computation of a \el{handle} or \el{elab} construct. This is important to ensure correct behaviour for when there are multiple nested handlers for a single effect. In that case, only the innermost handler should be able to handle said effect. The same reasoning applies to elaborations. We use the $\in$ symbol to mean that an operation is not handled by a handler, or elaborated by an elaboration.

\cref{fig:reductionsemantics} with semantics does not include semantics for declarations and modules. A sequence of declarations is evaluated in order. The bindings from each declaration get substituted in the remainder of the program. If the declarations are inside a module declaration, then the public bindings get collected. These bindings are substituted when a module is imported with the \el{use} declaration.

In this semantics, we assume that all elaborations are explicit. If they are implicit, they first need to be transformed according to the procedure from \cref{chap:elabres}.

\begin{figure}[p]
\begin{align*}
    E
        \IS & [] \OR E(e_1,\dots, e_n) \OR v(v_1,\dots,v_n,E,e_1,\dots,e_m) \\
        \OR & \cond{E}{e_1}{e_2} \\
        \OR & \kw{match}\;E\;\tok{\{}\;p_1\;\tok{=>}\;e_1\tok{,}\dots\tok{,}\;p_n\;\tok{=>}\;e_n\;\tok{\}} \\
        \OR & \kw{let}\;x = E;\;e \OR E;\;e\\
        \OR & \kw{handle}\tok{[}E\tok{]}\;e \OR \kw{handle}\tok{[}v\tok{]}\;E \\
        \OR & \kw{elab}\tok{[}E\tok{]}\;e \OR \kw{elab}\tok{[}v\tok{]}\;E\\
    \\
    X_{op}
        \IS & [] \OR X_{op}(e_1, \dots, e_n) \OR v(v_1, \dots, v_n, X_{op}, e_1, \dots, e_m) \\
        \OR & \cond{X_{op}}{e_1}{e_2} \\
        \OR & \kw{match}\;X_{op}\;\tok{\{}\;p_1\;\tok{=>}\;e_1\tok{,}\dots\tok{,}\;p_n\;\tok{=>}\;e_n\;\tok{\}} \\
        \OR & \kw{let}\;x = X_{op};\;e \OR X_{op};\;e \\
        \OR & \kw{handle}\tok{[}X_{op}\tok{]}\;e \OR \kw{handle}\tok{[}h\tok{]}\;{X_{op}} \text{ if } op\not\in h \\
        \OR & \elab\tok{[}X_{op}\tok{]}\;e \OR \kw{elab}\tok{[}\epsilon\tok{]}\;X_{op} \text{ if } op \not\in \epsilon
\end{align*}
\begin{align*}
    (\lambda x_1, \dots, x_n . e) (v_1, \dots, v_n) \reduce& e[x_1 \mapsto v_1, \dots, x_n \mapsto v_n] \\
    \cond{\true}{e_1}{e_2} \reduce& e_1 \\
    \cond{\false}{e_1}{e_2} \reduce& e_2 \\
    \begin{aligned}
    &\kw{match}\;c(v_1,\dots, v_n)\;\tok{\{}\\
    &\qquad\dots\\
    &\qquad c(x_1, \dots, x_n)\;\tok{=>}\;e\\
    &\qquad\dots\\
    &\tok{\}}
    \end{aligned}
    \reduce& e[x_1 \mapsto v_1, \dots, x_n \mapsto v_n]
    \\
    \kw{handle}\tok{[}h\tok{]}\;v \reduce& e[x\mapsto v] \\
    &\qquad\text{where } \return(x) \S{ e } \in h\\
    \kw{handle}\tok{[}h\tok{]}\;X_{op}[op(v_1, \dots, v_n)] \reduce& e[x_1\mapsto v_1, \dots, x_n\mapsto v_n, \kw{resume} \mapsto k] \\
    &\qquad \text{where } \begin{aligned}[t]
        & op(x_1, \dots, x_n) \S{e} \in h\\
        & k = \lam{y}{\kw{handle}\tok{[}h\tok{]}\;
        X_{op}[y]}
    \end{aligned}\\
    \kw{elab}\tok{[}\epsilon\tok{]}\;v \reduce& v\\
    \kw{elab}\tok{[}\epsilon\tok{]}\;X_{op!}[op!(e_1, \dots, e_n)] \reduce& \kw{elab}\tok{[}\epsilon\tok{]}\;X_{op!}[e[x_1 \mapsto e_1, \dots, x_n \mapsto e_n]] \\
    &\qquad \text{where } op!(x_1, \dots, x_n) \S{e} \in \epsilon \\
\end{align*}
\caption{Reduction semantics for Elaine.}
\label{fig:reductionsemantics}
\end{figure}

\section{Standard Library}\label{sec:std}

To simplify parsing, Elaine does not include any operators. For the lack of operators, any manipulation of primitives needs to be done via the standard library of built-in functions. These functions reside in the \el{std} module, which can be imported like any other module with the \el{use} statement to bring its contents into scope.

The full list of functions available in the \el{std} module, along with their signatures and descriptions, is given in \cref{fig:std}. The \el{std} module also contains several other modules that can be imported. These modules are all written in Elaine itself. An overview of these modules is given in \cref{fig:submods} with the signatures of the functions and types they contain.

\begin{figure}[htbp]
\begin{subfigure}{\textwidth}
\begin{tabular}{lllll}
& Name & Type signature &  & Description \\
\hline
Arithmetic
& \code{add}    & \code{fn(Int, Int)} & \code{Int} & addition \\
& \code{sub}    & \code{fn(Int, Int)} & \code{Int} & subtraction \\
& \code{neg}    & \code{fn(Int)}      & \code{Int} & negation \\
& \code{mul}    & \code{fn(Int, Int)} & \code{Int} & multiplication \\
& \code{div}    & \code{fn(Int, Int)} & \code{Int} & division \\
& \code{modulo} & \code{fn(Int, Int)} & \code{Int} & modulo \\
& \code{pow}    & \code{fn(Int, Int)} & \code{Int} & exponentiation \\
\hline
Comparisons
& \code{eq}  & \code{fn(Int, Int)} & \code{Bool} & equality \\
& \code{neq} & \code{fn(Int, Int)} & \code{Bool} & inequality \\
& \code{gt}  & \code{fn(Int, Int)} & \code{Bool} & greater than \\
& \code{geq} & \code{fn(Int, Int)} & \code{Bool} & greater than or equal \\
& \code{lt}  & \code{fn(Int, Int)} & \code{Bool} & less than \\
& \code{leq} & \code{fn(Int, Int)} & \code{Bool} & less than or equal \\
\hline
Boolean operations
& \code{not} & \code{fn(Bool)} & \code{Bool} & boolean negation \\
& \code{and} & \code{fn(Bool, Bool)} & \code{Bool} & boolean and \\
& \code{or} & \code{fn(Bool, Bool)} & \code{Bool} & boolean or \\
\hline
String operations
& \code{concat} & \code{fn(Bool, Bool)} & \code{Bool} & string concatenation \\
& \code{is_prefix} & \code{fn(String, String)} & \code{Bool} & is prefix of \\
& \code{str_eq} & \code{fn(String, String)} & \code{Bool} & string equality \\
& \code{drop} & \code{fn(Int, String)} & \code{String} & drop characters \\
& \code{take} & \code{fn(Int, String)} & \code{String} & take characters \\
& \code{length} & \code{fn(String)} & \code{Int} & string length \\
\hline
Conversions
& \code{show_int} & \code{fn(Int)} & \code{String} & integer to string \\
& \code{show_bool} & \code{fn(Bool)} & \code{String} & integer to string \\
\end{tabular}
\caption{The built-in functions in the \el{std} module in Elaine.}
\label{fig:std}
\end{subfigure}
\begin{subfigure}{\textwidth}
    \vspace{3em}
    \centering
    \begin{tabular}{lll}
    Module & Item & Type signature \\
    \hline
    \code{loop}
    & \code{while} & \code{fn(fn() <|e> Bool, fn() <|e> ()) <|e> ()} \\ 
    & \code{repeat} & \code{fn(Int, fn(Int) <|e> ()) <|e> ()} \\
    \hline
    \code{maybe}
    & \code{Maybe[a]} & \code{type Maybe[a] \{ Just(a), Nothing() \}} \\
    \hline
    \code{abort}
    & \code{abort} & \code{effect Abort \{ abort() a \}} \\
    & \code{hAbort} & \code{Handler Abort a Maybe[a]}\\
    \hline
    \code{list}
    & \code{List} & \code{type List[a] \{ Cons(a, List[a]), Nil() \}} \\
    & \code{head} & \code{fn(List[a]) Maybe[a]} \\
    & \code{concat_list} & \code{fn(List[a], List[a]) List[a]} \\
    & \code{range} & \code{fn(Int, Int) List[Int]} \\
    & \code{map} & \code{fn(fn(a) <|e> b, List[a]) <|e> List[b]} \\
    & \code{foldl} & \code{fn(fn(a, b) <|e> b, b, List[a]) <|e> List[b]} \\
    & \code{foldr} & \code{fn(fn(a, b) <|e> b, b, List[a]) <|e> List[b]} \\
    & \code{sum} & \code{fn(List[Int]) Int} \\
    & \code{join} & \code{fn(List[String]) String} \\
    & \code{explode} & \code{fn(String) List[String]} \\
    \hline
    \code{state}
    & \code{State} & \code{effect State \{ get() Int, put(Int) () \}} \\
    & \code{hState} & \code{Handler[State,a,fn(Int) a]} \\
    \hline
    \code{state_str}
    & \code{State} & \code{effect State \{ get() String, put(String) () \}} \\
    & \code{hState} & \code{Handler[State,a,fn(String) a]}
    \end{tabular}
    \caption{Sub-modules of \el{std}}
    \label{fig:submods}
\end{subfigure}
\caption{Overview of Elaine's standard library.}
\end{figure}
