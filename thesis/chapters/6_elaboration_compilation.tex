\chapter{Elaboration Compilation}\label{chap:elabcomp}

Since Elaine has a novel semantics for elaborations, it is worth examining its relation to well-studied constructs from programming language theory. Therefore, we introduce a transformation from programs with higher-order effects to a program with only algebraic effects, translating higher-order effects into algebraic effects, while preserving their semantics.

The goal of this transformation is twofold. First, it further connects hefty algebras and Elaine to existing literature. For example, by compiling to a representation with only algebraic effects, we can then further compile the program using existing techniques, such as the compilation procedures defined for Koka \autocite{leijen_type_2017}. In this thesis and the accompanying implementation, we provide the first step of this compilation. Second, the transformation allows us to encode elaborations in existing libraries and languages for algebraic effects.

\section{Non-locality of Elaborations}

\TODO{Actually it's not just non-locality but also undecidable}

Examining the semantics of elaborations, we observe that elaborations perform a syntactic substitution. For instance, the program on the left transforms into the program on the right by replacing \el{plus_two!}, with the expression \el{\{ x + 2 \}}.

\begin{minipage}[b]{0.5\textwidth}
\begin{lst}{Elaine}
use std;

effect PlusTwo! {
    plus_two!(Int)
}

let ePlusTwo = {
    elaboration PlusTwo! -> <> {
        plus_two!(x) { add(x, 2) }
    }
};

let main = elab plus_two!(5);
\end{lst}
\end{minipage}
\begin{minipage}[b]{0.5\textwidth}
\begin{lst}{Elaine}
let main = { add(5, 2) };
\end{lst}
\end{minipage}

Additionally, the location of the \el{elab} does not matter as long as the operations are evaluated within it. For instance, these expressions are equivalent:

\begin{minipage}[b]{0.5\textwidth}
\begin{lst}{Elaine}
let main = elab[e] {
    a!();
    a!()
};
\end{lst}
\end{minipage}
\begin{minipage}[b]{0.5\textwidth}
\begin{lst}{Elaine}
let main = {
    elab[e] a!();
    elab[e] a!()
};
\end{lst}
\end{minipage}

In some cases, it is therefore possible to statically determine the elaboration that should be applied. In that situation, we can remove the elaboration from the program by performing the syntactic substitution.

However, we cannot apply that technique in general. One example where it does not work is when the elaboration is given by a complex expression, such as an \el{if}-expression:
\begin{lst}{Elaine}
elab[if cond { elab1 } else { elab2 }] c
\end{lst}

Moreover, a single operation might need to be elaborated by different \el{elab} constructs, depending on run-time computations. In the listing below, there are two elaborations \el{eOne} and \el{eTwo} of an operation \el{a!()}. The \el{a!()} operation in \el{f} is elaborated where \el{f} is called. If the condition \el{k} evaluates to \el{true}, \el{f} is assigned to \el{g}, which is elaborated by \el{eOne}. However, if \el{k} evaluates to \el{false}, \el{f} is called in the inner \el{elab} and hence \el{a!()} is elaborated by \el{eTwo}.

\begin{lst}{Elaine}
elab[eOne] {
    let g = elab[eTwo] {
        let f = fn() { a!() };
        if k {
            f
        } else {
            f();
            fn() { () }
        }
    }
    g()
}
\end{lst}

Therefore, the analysis of determining the elaboration that should be applied to an operation is non-local. The static substitution could be used as an optimization or simplification step, but it cannot guarantee that the transformed program will not contain higher-order effects.

\section{Operations as Functions}

As explained in \cref{sec:semantics}, higher-order operations are evaluated differently from functions. The main difference is that the arguments are thunked and passed by name, instead of by value. 

This behaviour can be emulated for functions if anonymous functions are passed as arguments instead of expressions. That is, for any operation call \el{op!(e1, ..., eN)}, we wrap the arguments into functions to get \el{op(fn() \{ e1 \}, ..., fn() \{ eN \})}. In the body of \el{op}, we then replace each occurrence of an argument \el{x} with \el{x()} such that the thunked value is obtained. The \el{op} operation can then be evaluated like a function instead, but it still has the intended semantics.

\section{Compiling Elaborations to Dictionary Passing}

\TODO{This is currently just an example. The actual transformation needs to be clearly defined too.}

Instead, the elaborations can be transforms with a technique similar to dictionary-passing style: the implicit context of elaborations is explicitly passed to functions that require a higher-order effect.

Any function with higher-order effects then takes the elaboration to apply as an argument and the operation is wrapped in an \el{elab}. The elaboration is then taken from \el{elabA} at the call-site.

An example of what this transformation is given in \cref{lst:untransformed,lst:transformed}, where \cref{lst:untransformed} shows an untransformed program with higher-order effects and \cref{lst:transformed} shows the result of the transformation.

\begin{lstlisting}[language=elaine,style=fancy,float,caption={Untransformed program. \fixme{This example should use a higher-order effect.}},label={lst:untransformed}]
use std;

effect A! {
    arithmetic!(Int, Int) Int
}

let eAdd = elaboration A! -> <> {
    arithmetic!(a, b) { add(a, b) }
};

let eMul = elaboration A! -> <> {
    arithmetic!(a, b) { mul(a, b) }
};

let foo = fn(k) {
    elab[eAdd] {
        let g = elab[eMul] {
            let f = fn() { a!(5, 2 + 3) };
            if k {
                f
            } else {
                let x = f();
                fn() { x }
            }
        };
        g()
    }
};
\end{lstlisting}
\begin{lstlisting}[language=elaine,style=fancy,float,caption={Transformed program after compiling elaborations to dictionary passing.},label={lst:transformed}]
use std;

effect A! {
    arithmetic!(Int, Int) Int
}

let eAdd = elaboration A! -> <> {
    arithmetic!(a, b) { add(a, b) }
};

let eMul = elaboration A! -> <> {
    arithmetic!(a, b) { mul(a, b) }
};

# Create a new type with one constructor to represent the
# elaboration. The fields of the constructor are the 
# operations. We assume for convenience that all the
# generated identifiers do not conflict with existing
# identifiers.
type ElabA {
    ElabA(fn(fn() Int, fn() Int) Int)
}

# Convenience function to access the operation a from A!
let elab_A_a = fn(e: ElabA) {
    let ElabA(v) = e;
    v
};

let eAdd = ElabA ( fn(a, b) { add(a(), b()) } );
let eMul = ElabA ( fn(a, b) { mul(a(), b()) } );

# The transformed program
let foo = fn(k) {
    let elab_A = eAdd;
    let g = {
        let elab_A = eMul;
        let f = fn(elab_A) {
            elab_A_a(elab_a)(fn() { 5 }, fn() { 2 + 3 })
        };
        if k {
            f
        } else {
            let x = f(elab_A);
            fn(elab_A) { x }
        }
    };
    g(elab_A_a)
};
\end{lstlisting}

\section{Compiling Elaborations into Handlers}

\TODO{Talk about impredicativity and how that makes it so that it does not work.}

While the transformation in the previous section is correct, the transformed program is quite verbose, because the elaboration types need to be passed to every function with higher-order effects. It would be more convenient if this was passed implicitly.

As it turns out, we have a mechanism for passing implicit arguments: algebraic effects! Conceptually, both \el{elab} and \el{handle} are similar: they define a scope in which a given elaboration or handler is used. This scope is the same for both.

To use this observation, we start by defining a handler that returns an elaboration for higher-order effect \el{A!}, much like the \el{Ask} effect from \cref{chap:basics}.

Combining the ideas above, we obtain a surprisingly simple transformation. Each elaboration is transformed into a handler, which resumes with a function containing the original expression, where argument occurrences force the thunked values. Since elaborations are now handlers, we need to change the \el{elab} constructs to \el{handle} constructs accordingly. Finally, the arguments to operation calls are thunked and the function that is resumed is called, that is, there is an additional \el{()} at the end of the operation call.

\begin{figure}[H]
\begin{tabular}{rcl}
\begin{lstlisting}
elab[$e_1$] {$e_2$}
\end{lstlisting}
& $\implies$
& \begin{lstlisting}
handle[$e_1$] {$e_2$}
\end{lstlisting}
\\\\
\begin{lstlisting}
elaboration {
    $op_1!(x_{1,1}, ..., x_{k_1,1})$ { $e_1$ }
    ...
    $op_n!(x_{1,n}, ..., x_{k_n,n})$ { $e_n$ }
}
\end{lstlisting}
&$\implies$
&\begin{lstlisting}
handler {
    $op_1(x_{1,1}, ..., x_{k_1,1})$ {
        resume(fn() {$e_1$[$x_{i,1} \mapsto x_{i,1}()$]})
    }
    ...
    $op_n(x_{1,n}, ..., x_{k_n,n})$ {
        resume(fn() {$e_1$[$x_{i,n} \mapsto x_{i,n}()$]})
    }
}
\end{lstlisting}
\\\\
\begin{lstlisting}
$op_j$!($e_1,...,e_k$)
\end{lstlisting}
& $\implies$
& \begin{lstlisting}
$op_j$(fn() \{$e_1$\}, ..., fn() \{$e_k$\})()
\end{lstlisting}
\end{tabular}
\end{figure}

The simplicity of the transformation makes it alluring and begs the question: are dedicated language features for higher-order effects necessary or is a simpler approach possible?

\TODO{Answer: yes if you care about impredicativity, no otherwise.}
