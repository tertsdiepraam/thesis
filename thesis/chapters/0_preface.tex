\chapter{\label{chap:Preface}Preface}
\addcontentsline{toc}{subsection}{Preface}
Preface here.

\vspace{1cm}
\begin{flushright}
\theauthor{}\\
Delft, the Netherlands\\
\today{}\\
\end{flushright}

\vspace*{5em}
\paragraph{About the cover illustration} The cover image depicts Rubin's Vase in the shape of the holy grail. Rubin's Vase is a shape that is bi-stable, meaning that it can be viewed in two ways. One might see either the two faces or the vase. This represents how effectful computations in Elaine can be given different interpretations with different handlers and elaborations. This version of Rubin's Vase is meant to resemble the Holy Grail. After having chosen the name Elaine for the language developed for this thesis, we learned that several characters in Arthurian legend are called Elaine. One of these characters is Elaine of Corbenic, who is also called the ``Grail Bearer'' or the ``Grail Maiden''.
