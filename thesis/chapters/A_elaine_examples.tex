\chapter{Elaine Example Programs}\label{chap:examples}

This chapter contains longer Elaine samples with some additional explanation.

\section{A naive SAT solver}\label{sec:sat}

This program is a naive brute-forcing SAT solver. We first define a \el{Yield} effect, so we can yield multiple values from the computation. We will use this to find all possible combinations of boolean inputs that satisfy the formula. The \el{Logic} effect has two operations. The \el{branch} operation will call the continuation twice; once with \el{false} and once \el{true}. With \el{fail}, we can indicate that a branch has failed. To find all solutions, we just \el{branch} on all inputs and \el{yield} when a correct solution has been found and \el{fail} when the formula is not satisfied. In the listing below, we check for solutions of the equation $\neg a \wedge b$.

\example{logic}

\section{The Reader Effect}\label{sec:reader}

\TODO{explain}

\example{local_reader}

\section{Structured Logging}

\TODO{explain}

\example{structured_logging}

\section{Parser Combinators}

Monadic parser combinators \autocite{hutton_monadic_1996} are a popular technique for constructing parsers. The parser for Elaine is also written using \lib{megaparsec}{https://github.com/mrkkrp/megaparsec}, which is a monadic parser combinator library for Haskell. Attempts have been made to implement parser combinators using algebraic effects. However, it requires higher-order combinators for a full feature set matching that of monadic parser combinators. For example, the \hs{alt} combinator takes two branches and attempts to parse the first branch and tries the second branch if the first one fails. This is remarkably similar to the catch operation of the exception effect and is indeed higher-order.

Below is a full listing of a JSON parser written in Elaine using a variation on parser combinators using effects. It is implemented using a higher-order \el{Parse!} effect, which is elaborated into a state and an abort effect, which are imported from the standard library. The \el{try!} effect is a higher-order effect which takes an effectful computation as an argument.

Higher-order effects are convenient for parser combinators, but not necessary. There is a parsing effect in a more algebraic style written in Effekt available at \url{https://effekt-lang.org/docs/casestudies/parser} 

\example{parser_combinator}