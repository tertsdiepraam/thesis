\noindent
%In recent years, algebraic effects and handlers have been integrated into programming languages.
% Algebraic effects and handlers, an abstraction for effectful computation, have become more and more popular, with implementations in mainstream languages.
\emph{Algebraic effects} and \emph{handlers} have become a popular abstraction for effectful computation, with implementations even in mainstream programming languages, such as OCaml.
%
%Using algebraic effects, the effect operations and handlers can be declared separately, which provides modularity. 
%
%With algebraic effects, effect operations can be declared separately from their corresponding effect handlers. This provides modularity, because these effect handlers can be changed without changing the program.
%With algebraic effects, effect operations define an interface, which is implemented by effect handlers.
%The operations of an algebraic effect define an interface, which is implemented by an effect handler.
The operations of an algebraic effect define the syntax of the effect, while handlers define the semantics.
%
This provides modularity, because we can choose which handler to apply to a computation.
%
However, we cannot write handlers for many \emph{higher-order operations}; operations that take effectful computations as parameters. Such higher-order operations can therefore not enjoy this modularity.
%
\emph{Hefty algebras} provide an additional layer of abstraction in the form of \emph{elaborations} to make implementations of higher-order operations modular as well.
%
%At present, there are no programming languages with based on this theory yet.
Several languages, such as Koka, natively support algebraic effects and handlers.
%
% However, there are no languages with support for elaborations of higher-order effects.
However, until now, no languages have been created with native support for higher-order effects.
%
In this thesis, we introduce Elaine, a language featuring both handlers for algebraic effects and elaborations for higher-order effects.
%
%Additionally, we introduce the concept of implicit elaboration resolution, where the programmer does not have to specify which elaborations are applied, but the elaborations are instead inferred from the context.
Additionally, we introduce \emph{implicit elaboration resolution}; a type-directed procedure which infers the appropriate elaborations from context.
%
We conjecture that hefty algebras are the semantics for Elaine.
%
We provide a specification for Elaine, including its syntax definition, typing judgments and reduction semantics.
%
This specification is implemented in a publicly available prototype which can type check and evaluate the set of example programs included with this thesis.
%
