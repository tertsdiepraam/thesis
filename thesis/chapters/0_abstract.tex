\noindent In recent years, algebraic effects and handlers have been integrated into programming languages. This framework allows effect operations to be declared separately from their handlers, which provide the implementation. Hence, algebraic effects provide modularity. However, the techniques for algebraic effects can only be used for effect operations which satisfy the algebraicity property. Many higher-order operations, such as the catch operation of the except effect and the local operation of the reader effect, do not satisfy this property. These operations can therefore not be written as effects and are therefore not modular. The theory of hefty algebras has been proposed as a solution to this modularity problem. In this framework, modularity of higher-order operations is provided in the form of elaborations. However, there are no programming languages with based on this theory yet. In this thesis, we introduce Elaine, a language featuring both algebraic effect handlers and elaborations for higher-order effects. We provide a specification for Elaine including syntax definition, typing judgments and reduction semantics. We conjecture that hefty algebras are the semantics for Elaine. This specification is implemented in a prototype which can type check and evaluate a set of example programs included with this thesis. Additionally, we introduce the concept of implicit elaboration resolution, where the programmer does not have to specify which elaborations are applied, but the elaborations are instead inferred from the context.
