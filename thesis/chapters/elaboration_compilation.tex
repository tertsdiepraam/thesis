\chapter{Elaboration Compilation}\label{chap:elabcomp}

Since Elaine has novel semantics for elaborations, it is worth examining its relation to well-studied constructs from programming language theory. We therefore defined a transformation from elaborations into handlers, translating higher-order effects into algebraic effects, while preserving their semantics.

The goal of this transformation is twofold. First, it further connects hefty algebras and Elaine to existing literature. For example, by compiling to a representation with only algebraic effects, we can then further compile the program using existing techniques, such as the compilation procedures defined for Koka \autocite{leijen_type_2017}. Second, the transformation allows us to encode elaborations in existing libraries and languages for algebraic effects.

In this thesis and the accompanying implementation, we only provide the first step: transforming elaborations into handlers.

\section{Observations about Elaborations}

Examining the semantics of elaborations, we observe that elaborations are similar to macros: they perform a syntactic substitution. For instance, the program on the left transforms into the program on the right by replacing \el{plus_two!}, with the expression \el{\{ x + 2 \}}.

\begin{minipage}[b]{0.5\textwidth}
\begin{lstlisting}[language=elaine,style=fancy]
use std;

effect PlusTwo! {
    plus_two!(Int)
}

let ePlusTwo = {
    elaboration PlusTwo! -> <> {
        plus_two!(x) { add(x, 2) }
    }
};

let main = elab plus_two!(5);
\end{lstlisting}
\end{minipage}
\begin{minipage}[b]{0.5\textwidth}
\begin{lstlisting}[language=elaine,style=fancy]
let main = { add(5, 2) };
\end{lstlisting}
\end{minipage}

Additionally, the location of the \el{elab} does not matter as long as the operations are contained within it. For instance, these expressions are equivalent:

\begin{minipage}[b]{0.5\textwidth}
\begin{lstlisting}[language=elaine,style=fancy]
let main = elab[e] {
    a!();
    a!()
};
\end{lstlisting}
\end{minipage}
\begin{minipage}[b]{0.5\textwidth}
\begin{lstlisting}[language=elaine,style=fancy]
let main = {
    elab[e] a!();
    elab[e] a!()
};
\end{lstlisting}
\end{minipage}

In some cases, it should therefore be possible to match up elaborations and operations at compile-time. However, this does not hold in general. A clear hurdle is that we might have a complex expression for the elaboration, such as an \el{if} expression:
\begin{lstlisting}[language=elaine,style=fancy]
elab[if cond { elab1 } else { elab2 }] c
\end{lstlisting}
In this example, the elaboration might depend on a run-time value, making compile-time substitution difficult.

Moreover, a single operation might need to be elaborated by different \el{elab} constructs, depending on run-time computations. In the listing below, there are two elaborations \el{eOne} and \el{eTwo} of an effect with the operation \el{a!()}. The \el{a!()} operation in \el{f} is elaborated where \el{f} is called. If the condition \el{k} evaluates to \el{true}, \el{f} is assigned to \el{g}, which is elaborated by \el{eOne}. However, if \el{k} evaluates to \el{false}, \el{f} is called in the inner \el{elab} and hence \el{a!()} is elaborated by \el{eTwo}.

\begin{lstlisting}[language=elaine,style=fancy]
elab[eOne] {
    let g = elab[eTwo] {
        let f = fn() { a!() };
        if k {
            f
        } else {
            f();
            fn() { () }
        }
    }
    g()
}
\end{lstlisting}

Therefore, the analysis of determining the elaboration that should be applied to an operation is non-local. The substitution might still be a nice optimization or simplification step, but it cannot guarantee that the transformed program will not contain higher-order effects.

Instead, the elaborations can be localized with a technique similar to dictionary-passing style. Any function with higher-order effects then takes the elaboration to apply as an argument and the operation is wrapped in an \el{elab}. The elaboration is then taken from \el{elabA} at the call-site.
\begin{lstlisting}[language=elaine,style=fancy]
let elabA = eOne;
let g = {
    let elabA = eTwo;
    let f = fn(e) { elab[e] a!() };
    if k {
        f
    } else {
        f(elabA);
        fn() { () }
    }
}
g(elabA)
\end{lstlisting}

\section{Encoding Macros as Functions}
\TODO{Macros are functions with arguments and return value transformed wrapped in functions.}

The main difference between a function and a macro-like operation in our semantics is that macro arguments are thunked. This holds because the macro is type-checked independently and is not allowed to use anything apart from its arguments.\question{Macro is not really the right word, but it is macro-like. What term would be good to use?} Hence, we can translate the higher-order operations to functions, where the functions are defined by the elaboration.

For any operation call \el{op!(e1, ..., eN)}, we wrap the arguments into functions to get \el{op!(fn() \{ e1 \}, ..., fn() \{ eN \})}. In the body of \el{op!}, we then replace each occurrence of an argument \el{x} with \el{x()} such that the thunked value is obtained. The \el{op!} operation can then be evaluated like a function instead of having special semantics for its evaluation.

\section{Handlers as Dictionary Passing}

We now have two pieces of the puzzle: we know how to translate the semantics for higher-order operations into regular functions, and we have determined that we need something like dictionary passing to determine what elaboration should be applied. The problem is that our language -- as currently defined -- does not support dictionaries. Even if dictionaries were defined, the type safety of the transformed program would be difficult to check without a sub-structural type system.

This leads to the final observation: effect handlers can be used as our dictionary passing construct. Conceptually, both \el{elab} and \el{handle} work similarly: they define a scope in which a given elaboration or handler is used. This scope is the same for both. We start by defining a handler that returns an elaboration for higher-order effect \el{A!}, much like the \el{Ask} effect from \cref{chap:basics}.

\begin{lstlisting}[language=elaine,style=fancy]
let hElab = fn(e) {
    handler {
        return(x) { x }
        askElabA() { resume(e) }
    }
};

handle[hElab(eOne)] {
    let g = handle[hElab(eTwo)] {
        let f = fn(e) { elab[askElabA()] a!() };
        if k {
            f
        } else {
            f();
            fn() { () }
        }
    }
    g()
}
\end{lstlisting}

\section{Compiling Elaborations into Handlers}

Combining the ideas above, we obtain a surprisingly simple transformation. Each elaboration is transformed into a handler, which resumes with a function containing the original expression, where argument occurrences force the thunked values. Since elaborations are now handlers, we need to change the \el{elab} constructs to \el{handle} constructs accordingly. Finally, the arguments to operation calls are thunked and the function that is resumed is called, that is, there is an additional \el{()} at the end of the operation call.

\begin{figure}[H]
\begin{tabular}{rcl}
\begin{lstlisting}
elab[$e_1$] {$e_2$}
\end{lstlisting}
& $\implies$
& \begin{lstlisting}
handle[$e_1$] {$e_2$}
\end{lstlisting}
\\\\
\begin{lstlisting}
elaboration {
    $op_1!(x_{1,1}, ..., x_{k_1,1})$ { $e_1$ }
    ...
    $op_n!(x_{1,n}, ..., x_{k_n,n})$ { $e_n$ }
}
\end{lstlisting}
&$\implies$
&\begin{lstlisting}
handler {
    $op_1(x_{1,1}, ..., x_{k_1,1})$ {
        resume(fn() {$e_1$[$x_{i,1} \mapsto x_{i,1}()$]})
    }
    ...
    $op_n(x_{1,n}, ..., x_{k_n,n})$ {
        resume(fn() {$e_1$[$x_{i,n} \mapsto x_{i,n}()$]})
    }
}
\end{lstlisting}
\\\\
\begin{lstlisting}
$op_j$!($e_1,...,e_k$)
\end{lstlisting}
& $\implies$
& \begin{lstlisting}
$op_j$(fn() \{$e_1$\}, ..., fn() \{$e_k$\})()
\end{lstlisting}
\end{tabular}
\end{figure}

\section{An Alternative Design for Elaine}

The simplicity of the transformation makes it alluring and begs the question: are dedicated language features for higher-order effects necessary or is a simpler approach possible?

Since we can encode elaborations as handlers, we can write higher-order effects as the result of the transformation above directly. Below is an example of an exception effect written in this style.

\begin{lstlisting}[language=elaine,style=fancy]
effect Exc {
    catch(fn() <Throw> a, fn() a)
}

let hExc = handler {
    return(x) { x }
    catch(f, g) {
        resume(fn() {
            let res = handle[hThrow] f();
            match res {
                Just(x) => x,
                Nothing => g(),
            }
        })
    }
};

let main = handle[hCatch] catch(
    fn() { ... },
    fn() { ... },
)();
\end{lstlisting}

Now let us introduce a few (hypothetical) syntactic conveniences. First, like Koka, we let \el{\{...\}}| represent \el{fn()\{...\}}. Second, we allow the \el{return} case to be omitted and default to the identity function. Third, if an operation ends with \el{!}, the operation body is wrapped in \el{resume(\{...\})} and the operations is given an underscore prefix with the original name brought into scope defined as

\begin{lstlisting}[language=elaine,style=fancy]
let op! = fn($x_1$, ..., $x_n$){ _op!($x_1$, ..., $x_n$)() };
\end{lstlisting}

Our previous example then reduces to

\begin{lstlisting}[language=elaine,style=fancy]
effect Exc {
    catch!(fn() <Throw> a, fn() a)
}

let hExc = handler {
    catch!(f, g) {
        let res = handle[hThrow] f();
        match res {
            Just(x) => x,
            Nothing => g(),
        }
    }
};

let main = handle[hExc] catch!(
    { ... },
    { ... },
);
\end{lstlisting}

The result is almost as convenient as the version of Elaine presented in the previous chapters and does not have the split between elaborations and handlers, which possibly makes it easier to understand. The same design could be applied to other implementations of algebraic effects as well with relative ease.

To illustrate that point, below is modular \el{catch} effect in Koka. While the handler is certainly more verbose than handlers for algebraic effects in Koka, the implementation is also quite simple. Koka could implement a shorthand for creating the boilerplate and essentially the same functionality as Elaine.

\begin{lstlisting}[language={},style=fancy]
effect abort
  ctl abort(): a
  
effect exc
  fun catch_( f : () -> <abort|e> a, g : () -> e a ) : (() -> e a)
  
val hAbort = handler
  return(x)   Just(x)
  ctl abort() Nothing

val hExc = handler
  fun catch_(f, g)
    fn()
      match hAbort(f)
        Just(x) -> x
        Nothing -> g()

fun catch(f, g)
  catch_(f, g)()

fun main()
  with hExc
  println(catch({ if True then abort() else 5 }, { 0 }))
\end{lstlisting}

Because we can then ``emulate'' higher-order effects, the importance for explicit support for them is reduced. An advantage of this approach is that, because we do not have to elaborate into algebraic effects, we do not have to handle them. In Elaine the \el{main} function looks like this:

\begin{lstlisting}[language=elaine,style=fancy]
let main = handle[hAbort] elab catch!(if true { abort() } else { 5 }, { 0 });
\end{lstlisting}

\TODO{Figure out whether this breaks down in Agda. Either way it needs more explaination.}
