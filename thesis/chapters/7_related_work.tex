\chapter{Related Work}\label{chap:related_work}

This chapter discusses some extensions to algebraic effects, some alternatives to algebraic effects. Additionally, we discuss some other languages with effects and the various alternative syntax and semantics.

\section{Monad Transformers}\label{sec:monad_transformers}

Monad transformers provide a way to compose monads \autocite{moggi_abstract_1989}. This makes them an alternative to the free monad. While monad transformers predate algebraic effects, they do support higher-order effects. A popular implementation of monad transformers is Haskell's mtl library. In the rest of this section, we adopt the terminology from that library.

The goal of monad composition is to make the operations of all composed monads available to the computation. Given two monads \hs{A} and \hs{B}, a naive composition would result in the type \hs{A (B a)}. However, this type represents a computation using \hs{A} that returns a computation \hs{B a}, meaning that it is not possible to use operations of both monads.

A monad transformer is a type constructor that takes some monad and returns a new monad. Usually, the transformation it performs is to add operations to the input monad. Composing \hs{A} and \hs{B} then requires some transformer \hs{AT} to be defined, such \hs{AT B} is a monad that provides the operations of both \hs{A} and \hs{B}. An arbitrary number of monad transformers can be composed this way. The representation of a monad then becomes much like that of a list of monad transformers. The \el{Identity} monad marks the end of the list, and it is defined as below.

\begin{lst}{Haskell}
newtype Identity a = Identity a
\end{lst}

\noindent A neat property of monad transformers is that a monad can be easily obtained by applying the transformer to the identity monad. Haskell's mtl library, for instance, defines a monad transformer \hs{StateT} and then defines \hs{State} as \hs{StateT Identity}. The operations of the state effect are then not implemented on \hs{StateT} directly, but on are part of a type class \hs{MonadState}. The \hs{StateT} is then an instance of \hs{MonadState} class. Every other transformer is an instance of \hs{MonadState} if its input monad is an instance of \hs{MonadState}. For example, for the \hs{WriterT} instance, there is the following instance declaration.

\begin{lst}{Haskell}
instance MonadState s (StateT s m) where
  -- definitions omitted

instance MonadState s m => MonadState s (WriteT m) where
  -- definitions omitted
\end{lst}

\noindent A computation can then be generic over the monad transformers, requiring only that \hs{StateT} is present somewhere in the stack of monad transformers.

\begin{lst}{Haskell}
usesState :: MonadState Int m => Int -> m Int
usesState a = get >>= \x -> put (x + a)
\end{lst}

\noindent This is analogous to the \hs{State s < f} constraint from the free monad encoding. However, there is a cost to this approach. For every effect, a new type class needs to be introduced and there need to be instance definitions on all existing monad transformers. The number of instance declarations therefore scales quadratically with the number of effects.

Another downside to monad transformers is that the order in which the monads need to be evaluated is entirely fixed. In the free monad encoding and languages with algebraic effects, the effects in the effect row can be reordered. \fixme{The order of the monad transformers also determines the order in which they must be handled: the outermost monad transformer must be handled first}\feedback{translate handling to the context of monad transformers}.

\section{Other Solutions to the Modularity Problem}

An alternative to hefty algebras for solving the modularity problem is the theory of \emph{scoped effects} \autocite{wu_effect_2014,pirog_syntax_2018,yang_structured_2022}. This theory replaces the free monad by a \hs{Prog} monad, which features one additional constructor called \hs{Enter}. Along with the continuation, this constructor takes a sub-computation. The return value of this sub-computation is passed to the continuation. In that sense, the \hs{Enter} constructor matches the \hs{>>=}, but without distributing the continuation over its sub-computation.

Instead of defining evaluation as a single algebra, scoped effects requires two algebras: an endo-algebra for scoped operations and a base-algebra for the other operations. This is somewhat similar to the distinction between elaboration and handling for hefty algebras, however, in hefty algebras, the algebras are not applied at the same time.

Many higher-order effects, such as the exception and reader effects, can be expressed in this framework. However, it is less general than hefty algebras, because there are some higher-order effects that cannot be expressed as scoped effects. This concerns effects that defer some computation, such as the lambda abstraction \autocite{oh_latent_2021}. Hefty algebras are therefore more general than scoped effects \autocite{bach_poulsen_hefty_2023}.

The limitations of scoped effects can be seen when we emulate them in Elaine. We can emulate the endo-algebra with a \el{handle} in the elaboration. Since the result of the sub-computation is directly passed to the continuation, the elaboration contains only a \el{handle} and nothing else. Therefore, any higher-order effect that can be expressed as the elaboration below (up to renaming) can be defined as a scope effect.

\begin{lst}{Elaine}
effect ScopedEffect! {
    scoped_operation!(a) a
}

let e = elaboration ScopedEffect! -> AlgebraicEffect {
    scoped_operation!(a) {
        handle[hAlg] a
    }
};
\end{lst}

\noindent Another alternative to hefty algebras are \emph{latent effects} \autocite{oh_latent_2021}, which supports the same set of effects as hefty algebras. \textcite{bach_poulsen_hefty_2023} argue that hefty algebras are a simpler model for higher-order effects than latent effects.

\section{Languages with First-Class Effects}
\TODO{Define first-class effects much earlier in the thesis}

The motivation of adding support for effects to a programming language is twofold. First, it enables effects to be implemented into languages with type systems in which effects cannot be encoded as a free monad or a similar model. Second, built-in effects allow for more ergonomic and performant implementations. Naturally, the ergonomics of any given implementation are subjective, but we undeniably have more control over the syntax by adding effects to the language.

Notable examples of languages with first-class support for algebraic effects are Eff \autocite{bauer_programming_2015}, Koka \autocite{leijen_koka_2014}, OCaml\citationneeded, and Frank \autocite{lindley_be_2017}. In all of these languages, effect row variables can be used to abstract over effects. For example, the signature of the \el{map} function is in Koka is given below and is similar to the signature of \el{map} in Elaine.
%
\begin{lst}{Koka}
fun map ( xs : list<a>, f : a -> e b ) : e list<b>
    ...
\end{lst}
%
Other languages choose a more implicit syntax for effect polymorphism. Frank \autocite{lindley_be_2017} opts to have the empty effect row represent the \emph{ambient effects}. Any effect row then becomes not the exact set of effects that need to be handled, but the smallest set. The equivalent signature of \code{map} is then written as
%
\begin{lst}{Frank}
map : {X -> []Y} -> List X -> []List Y
\end{lst}
%
In contrast with Elaine, languages such as Koka and Frank do not have dedicated types for handlers and \el{handle} constructs. Instead, they represent handlers as functions that take computations as arguments. In Elaine, there are dedicated types and constructs for effect handlers so that they are symmetric with elaborations. That is, the counterpart of \el{elab} is \el{handle} and the counterpart of \el{elaboration} is \el{handler}.

Koka implements several extensions to standard algebraic effects. First, it supports named handlers \autocite{xie_first-class_2022}, which provide a mechanism to distinguish between multiple occurrences of an effect in an effect row. Additionally, Koka features \emph{scoped handlers}, which are different from the previously mentioned scoped effects. Scoped handlers make it possible to associate types with handler instances \autocite{xie_first-class_2022}.

\section{Effects as Free Monads}

There are many libraries that implement the free monad in various forms in Haskell, including \lib{fused-effects}{https://github.com/fused-effects/fused-effects}, \lib{polysemy}{https://github.com/polysemy-research/polysemy}, \lib{freer-simple}{https://github.com/lexi-lambda/freer-simple} and \lib{eff}{https://github.com/hasura/eff}. Each of these libraries give the encoding of effects a slightly different spin in an effort to find the most ergonomic and performant representation. They are all not just based on the free monad, but on freer monads \autocite{kiselyov_freer_2016} and fused effects \autocite{hinze_fusion_2015} for better performance. Some of these libraries support scoped effects as well, but apart from the work by \textcite{bach_poulsen_hefty_2023}, no libraries with support for hefty algebras have been published.

Effect rows are often constructed using the \emph{Data Types à la Carte} technique \autocite{swierstra_data_2008}, which requires a fairly robust type system. Hence, many languages cannot encode effects within the language itself. In some languages, it is possible to work around the limitations with metaprogramming, such as the Rust library \lib{effin-mad}{https://github.com/rosefromthedead/effing-mad}, though the result does not integrate well with the rest of language and its use in production is strongly discouraged by the author.

The programming language Idris \autocite{brady_programming_2013} also has an implementation of algebraic effects in its standard library. It is an interesting case study since Idris is a dependently typed language. Due to its dependent typing, it can distinguish multiple occurrences of a single effect in the same effect row by assigning them different \emph{labels}. This is similar to what \emph{named handlers} \autocite{xie_first-class_2022} aims to accomplish.
